\chapter{Related works}
\label{chap:related}

\begin{itemize}

\item topic modeling, in general \autocite{topicModel}, by taking space into
account \autocite{GeoTopicYin11, GeoTopicKurashima2013, nestedChinese13,%
NonGaussianTopicKling14} as well as time \autocite{GeoScope, TwitterBurst13}

Mining spatial data to discover places and their semantics. This places are
usually defined by their density, and thus one seeks to find high concentration
of activity. For instance, \textcite{Deng2009} use DBSCAN to cluster Flickr
photos and then look at tags co occurrence within clusters to identify their
meaning. \Textcite{Rattenbury2009} also employ various methods to discover
regions in San Francisco where one photos tag appears significantly. The shape
of these hotspot regions can be refined by looking at the orientation of each
photos \autocite{Hotspots12}. Another way of identify hotspots is second order
Ripley's K-function \autocite{TagHotspot12}. Basically, it is a statistical
test for the hypothesis that points are distributed according to a Poisson
process and it tells how they \enquote{interact with each other, either
\enquote{repulsively} [\dots] or \enquote{attractively}.} Furthermore, it is
able to deal with varying density of pictures.

Moving from Flickr to \glspl{lbsn} allows to search for larger areas like
neighborhoods. \Textcite{SocioMap12} collect general tweets and perform
sentiment analysis as well as movement detection to draw a socio cognitive map
of the Kinki region in Japan which helps user to make decision regarding where
to live. Closer to our work, \textcite{Livehoods12} make use of 18 million
check-ins to find so-called
\emph{Livehoods}\footnote{\href{http://livehoods.org/}{\url{livehoods.org}}}.
They build a $m$ spatial neighbors graph of venues with edges weighted by the
cosine similarity between their user distribution and then perform spectral
clustering. Faced with the same difficulty to evaluate their results, they
interview residents of Pittsburgh who validate the obtained representation.
Another approach was proposed by
\textcite{Hoodsquare13}\footnote{\href{http://pizza.cl.cam.ac.uk/hoodsquare/}%
{\url{hoodsquare.org}}}, also based on Foursquare check-ins.  Each
venues is given its category, its peak time activity and a binary
label: touristic or not. They are clustered in hotspots along each of
these features by OPTICS. The city is then divided into a regular grid
and cells are described by its hotspot density for each feature.
\marginpar{Mention evaluation?} Finally, cells are clustered into neighborhoods
by their smoothed cosine similarity. The main difference between this two last
work and ours is that they do not explicitly mention distance between the
neighborhoods they recover.

\item smart cities \autocite{HelsinkiSCC11, Eunoia13, SmartCities13} (or
find nicest path within cities \cite{Quercia2014}
playable cities: \url{http://www.watershed.co.uk/playablecity/overview/},
\ie{} avoid focusing solely on monetization:
\url{http://radar.oreilly.com/2014/05/most-of-what-we-need-for-smart-cities-already-exists.html})

Even though I did not find publications, some companies are also using data to
better understand cities. For instance, Flickr has shown that by computing the
$\alpha$-shape of a set of tagged photos (which is a generalisation of its
convex hull \autocite{AlphaShape83}) they can recover boundaries of
neighborhoods: \href{http://code.flickr.net/2008/10/30/the-shape-of-alpha/}%
{\url{code.flickr.net/2008/10/30/the-shape-of-alpha}}.  Likewise, Airbnb, the
social lodging renting website, has accumulated a lot of spatial data as well
as textual reviews, which can be used to rank cities by hospitality
\href{http://nerds.airbnb.com/most-hospitable-cities/}%
{\url{nerds.airbnb.com/most-hospitable-cities}} as well as discovering
neighborhoods
\href{https://www.airbnb.com/locations}{\url{airbnb.com/locations}}.

\item learning from multi source \url{http://www.hiit.fi/cosco/mupi} (although
Flickr and Foursquare are not so different)

\end{itemize}

In addition to that,
% I read some papers about \textbf{metric learning} and
there must some about \textbf{matching objects}, even if the applications are
different.
