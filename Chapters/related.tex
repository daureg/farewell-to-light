\chapter{Related work}
\label{chap:related}

Our work can be related to several lines of research within the field of urban
computing. First, mining spatial data (photos, check-ins, and GPS
trajectories) to find places and their semantics, which is a potential
application of the broad term of \enquote{smart city} and has very concrete
implications for major websites. This task also remains of topic modeling,
with more constraints regarding space and time. After finding places, we want
to learn metrics to compare them in a meaningful and accurate way.

\paragraph{photos}

Such places are usually defined by their density, and thus one seeks to find
high concentration of people taking photos there. For instance,
\textcite{Deng2009} use \textsc{DBSCAN} to cluster Flickr photos and then look
at tags co occurrence within clusters to identify their meaning.
\Textcite{Rattenbury2009} also employ various spatial methods to discover
regions in San Francisco where one photo tag appears in burst. The shape of
these hotspot regions can be refined by looking at the orientation of each
photo \autocite{Hotspots12}. Another way of identify hotspots is second order
Ripley's K-function \autocite{TagHotspot12}. Basically, it is a statistical
test for the hypothesis that points are distributed according to a Poisson
process. It indicates how they \enquote{interact with each other, either
\enquote{repulsively} [\dots] or \enquote{attractively}.} Furthermore, it is
able to process regions with varying density of pictures. Besides location
finding, analyzing photos can also lead to other applications, like
recommending routes for tourists based on a graph of points of interest
\autocite{CityItineraries10} and even ensure that these paths are as agreeable
as possible \autocite{Quercia2014}.

\paragraph{check-ins}

Moving from Flickr to \glspl{lbsn} allows searching for larger areas like
neighborhoods. \Textcite{SocioMap12} collect general tweets and perform
sentiment analysis as well as movement detection to draw a socio cognitive map
of the Kinki region in Japan which helps user to make decision regarding where
to live. Closer to our work, \textcite{Livehoods12} analyse 18 millions
check-ins to find so-called
\emph{Livehoods}\footnote{\href{http://livehoods.org/}{\url{livehoods.org}}}.
They build a $m$ spatial neighbors graph of venues with edges weighted by the
cosine similarity between their user distribution and then perform spectral
clustering. Faced with the same difficulty as ours to evaluate their results,
they interview residents of Pittsburgh who validate the obtained subdivisions.
Another approach was proposed by
\textcite{Hoodsquare13}\footnote{\href{http://pizza.cl.cam.ac.uk/hoodsquare/}%
{\url{hoodsquare.org}}}, also based on Foursquare check-ins. Each venue is
described by its category, its peak time activity and a binary label:
touristic or not. They are clustered in hotspots along each of these dimensions
by the \textsc{OPTICS} algorithm. The city is then divided into a regular grid
and cells are described by their hotspot density for each feature.
\marginpar{Mention evaluation using tweets mention?} Finally, similar cells
are clustered into neighborhoods. The main difference between this last two
works and ours is that they do not explicitly mention distance between the
neighborhoods they recover.

\paragraph{GPS}

In addition to static data like photos, another line of work takes advantage
of the dynamic nature of human movement to mine trajectories. For instance,
\textcite{MAtlas11} analyze GPS data in Italian cities to find temporal
patterns, which can then be used for event detection or traffic jam
regulation. Similarly, \textcite{GPSStay10} extract stay points from car GPS
data and assess their significance by how many visitors go there, how far they
traveled to reach them and how long they stay. \Textcite{TrajROI11} describe
efficient techniques to perform closely related tasks. Once the semantics of
locations is known, it is still challenging to find frequent patterns
efficiently \autocite{SemanticTrajectories14}. Interested reader will find
more information in a recent survey \autocite{TrajSurvey13}.


\paragraph{Industrial applications}

The problem of identifying and characterizing neighborhoods has also been
addressed by companies. For instance, research in \flickr{} has shown that by
computing the $\alpha$-shape of a set of tagged photos (which is a
generalisation of its convex hull \autocite{AlphaShape83}) one can recover
boundaries of
neighborhoods\footnote{\href{http://code.flickr.net/2008/10/30/the-shape-of-alpha/}%
{\url{code.flickr.net/2008/10/30/the-shape-of-alpha}}}. Likewise, Airbnb, a
social lodging renting website, has accumulated a lot of spatial data as well
as textual reviews, which can be used to rank cities by hospitality
\footnote{\href{http://nerds.airbnb.com/most-hospitable-cities/}%
{\url{nerds.airbnb.com/most-hospitable-cities}}} as well as discovering
neighborhoods
\footnote{\href{https://www.airbnb.com/locations}{\url{airbnb.com/locations}}}.
However, the methods and details of these commercial systems are not publicly
available. Finally, engineers at \fs{} used 1.5 billion check-ins to compare
activities of different part of cities in
US\footnote{\href{http://engineering.foursquare.com/2012/03/08/a-hackday-project-what-neighborhood-is-the}%
{\url{engineering.foursquare.com/2012/03/08/a}}}.

\paragraph{Spatio-temporal topic modeling}

Finding such places can also be thought as trying \enquote{to discover the
hidden thematic structure in large archives of documents}, which is the
definition of probabilistic topic modeling according to \textcite{topicModel}.
Usually such documents are textual and can be combined in any way to form a
theme. Here we do not necessarily have text (although photos tags and check-ins
contain some) but we have additional spatio-temporal constraints for
neighborhoods (or topic) to be localized. For instance,
\textcite{GeoTopicYin11} fix a number $K$ of localized topics to be discovered,
as well as a set of $N$ weighted Gaussian spatial regions. Then they apply
their Latent Geographical Topic Analysis framework: each region has a topic
distribution and each topic is a multinomial distribution over all possible
photos tags. The parameters of the model are learned by an EM algorithm. By
taking user into account, this approach provides recommendations
\autocite{GeoTopicKurashima2013}. The same can be done for localized tweets
with a hierarchical model of topics \autocite{nestedChinese13}. Recently,
precision and running time of such methods were improved by
\textcite{NonGaussianTopicKling14}. Finally, some topics are localized in time
as well as in space (like \texttt{earthquake}) and this calls for more specific
methods. \Textcite{TwitterBurst13} extract keywords from twitter in a moving
sliding windows and keep those that are bursty. Using localized tweets, they
estimate the topic spatial distribution but because of sparsity and noise, it
requires a regularisation based on a co occurrences and identification of
anchor keywords. A scalable online approximate solution is described in
\autocite{GeoScope}.

\paragraph{Smart cities}

Smart cities in general are a new area which offers a lot of scientific
challenges \autocite[Chapter 4]{Eunoia13} as well as potential benefits in
terms of sustainability and well being by making use of Information
Technologies \autocite{SmartCities13}. How IBM was able to tap into Helsinki
open data to provide a compelling narrative of citizens' engagement and promote
the World Design Capital event in 2012 \autocite{HelsinkiSCC11}.

% (or playable cities: \url{http://www.watershed.co.uk/playablecity/overview/},
% \ie{} avoid focusing solely on monetization:
% \url{http://radar.oreilly.com/2014/05/most-of-what-we-need-for-smart-cities-already-exists.html})

\paragraph{Metric learning}

Some pointers we already given in \autoref{sec:metric} but the topic is
abundant. For instance, learning a Mahalanobis distance was introduced more
than ten years ago \autocite{Xing2002}. Later, the optimization problem was
given a simpler formulation, which has therefore a lower complexity
\autocite{Shen2011}. Two recent works are more sophisticated and use two-stage
statistical approach \autocite{Wang2014} and sparse local metrics
\autocite{SparseMetric14}.

\iffalse
\paragraph{Matching and multiple domains}

\begin{comments}
	So far, I haven't really found a connection between these papers and my work
	so I will probably remove them.
	\begin{itemize}
		\item 2 domains with the same number of objects in each (but described
			by different features) \autocite{Klami2013}.
		\item many to many matching from arbitrary number of domain and
			different number of object in each by using a common latent space
			\autocite{iwata2013unsupervised}
		\item Give unknown $K$ sources of observations and their pairwise
			dissimilarity, find the optimal set of sources to learn a model
			for each one \autocite{Crammer2008}.
		\item Train a classifier in a source domain where there are a lot
			labeled examples and see how well it would perform on close target
			domain with less examples (and how to use those)
			\autocite{Ben-David2009}.
	\end{itemize}
\end{comments}
\fi
