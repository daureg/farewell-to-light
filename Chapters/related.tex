\chapter{Related works}
\label{chap:related}

\begin{itemize}

\item topic modeling, in general \autocite{topicModel}, by taking space into
account \autocite{GeoTopicYin11, GeoTopicKurashima2013, nestedChinese13,%
NonGaussianTopicKling14} as well as time \autocite{GeoScope, TwitterBurst13}

% spatial mining (food?)

\item flickr data and hotspot \autocite{TagHotspot12, Hotspots12}

\item Searching explicitly for neighborhoods
\autocite{Rattenbury2009, Livehoods12, SocioMap12, Hoodsquare13}

\item smart cities \autocite{HelsinkiSCC11, Eunoia13, SmartCities13} (or
find nicest path within cities \cite{Quercia2014}
playable cities: \url{http://www.watershed.co.uk/playablecity/overview/},
\ie{} avoid focusing solely on monetization:
\url{http://radar.oreilly.com/2014/05/most-of-what-we-need-for-smart-cities-already-exists.html})

% transportation data

% \item graph and community detection (instead of clustering used in several
% places \autocite{PositinalCluster14})

\item non academic initiatives, flickr alpha shape
\url{http://code.flickr.net/2008/10/30/the-shape-of-alpha/}, airbnb
neighborhoods \url{https://www.airbnb.com/locations}

\item learning from multi source \url{http://www.hiit.fi/cosco/mupi} (although
Flickr and Foursquare are not so different)

\end{itemize}

In addition to that,
% I read some papers about \textbf{metric learning} and
there must some about \textbf{matching objects}, even if the applications are
different.
