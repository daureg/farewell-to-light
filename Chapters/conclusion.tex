\chapter{Discussion}
\label{chap:discussion}

\begin{flushright}{\slshape    
		The future is already here \\
		it's just not very evenly distributed.
    } \\ \medskip
    --- William Gibson
\end{flushright}

\section{Quality of results}

It's a difficult problem because the query is not clearly defined and
potentially convey high level semantics.

\section{Conclusion}

In this thesis, we define the problem of matching neighborhoods across cities
and introduce a method to solve it efficiently. We start by crawling social
networks for traces of geo-located human activity.
FS FL what how many
explo few salient observation
Selecting venues as the
elementary constituents of a city, we characterize them by features such as
time of activity, diversity of audience, popularity and repartition of
surrounding venues. We then evaluate several measures of similarity between
notable metric learning
venues and conclude that, using \fs\ categories as a proxy, \eucl{} and \lmnn{}
are the most suitable. Next, we consider various ways of measuring distance
between sets of venues and based on ground truth collected through a user
study, we find out that \emd{} outperforms other candidates. A exhaustive \emd\
search comes with the cost of many unnecessary calls of the distance function.
Therefore, we present a method to prune the search space which greatly enhance
run time while incurring only minor loss of accuracy compared with brute force
search.

\section{Future work}
\begin{itemize}
	\item Apply the same method to other problems (maybe for some where the ground truth is easier to define and to obtain).
	\item Use more data sources to derive more features (for instance
		transportation, weather, air quality, energy consumption, demographics,
		etc) More data trumps better algorithm \autocite{MoreData09}.
	\item Use the ground truth to optimize \emd{} and its ground		metric~\cite{LearnEMD14}.
	\item Automatically identify neighborhoods as dense areas sharing similar characteristics.
        \item Match several (or all) neighborhoods at the same time (but without excessive overlapping)
	\item Deliver interpretable results. Beyond a single number, the
		similarity, or absence thereof, must be motivated. It enables
		principled comparison of cities and more user friendly output (as well
		as justifying why no sensible result were found for some queries).
		Furthermore, this would address recent concerns about Algorithmic
		Accountability \autocite{Accountability13}. For instance,
		\textcite{Discrimination13} showed that in case of online advertisement
		discrimination, opacity of the algorithms makes it difficult to find
		and sanction those responsible.

		The easiest way to do it would be to find justification after the
		matching process (by clustering venues and summarizing it like
		\enquote{These two regions match because people take a lot of photos,
		go there on weekday from 4pm to 8pm and there are a lot cultural
		buildings.} Another source of information is the resulting flow of
		EMD. Looking at what kind of venues are close between two regions also
		provide insights. Yet it would be more satisfactory if the whole
		process was driven by such considerations instead of being a mere post
		processing step.
\end{itemize}

