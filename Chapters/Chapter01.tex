\chapter{Introduction}
\label{ch:introduction}

\section{Motivation}

An increasing share of the world population is living in
cities\footnote{\enquote{Between 2011 and 2050, the world population is
		expected to increase by 2.3 billion, passing from 7.0 billion
		to 9.3 billion. At the same time, the population living in
urban areas is projected to gain 2.6 billion, passing from 3.6 billion in 2011
to 6.3 billion in 2050.},
\href{http://esa.un.org/unup/pdf/WUP2011_Highlights.pdf}{UN report}.}.
Meanwhile, the rise of ubiquitous computing lets urban dwellers share
spatio-temporal information at an unprecedented scale
\autocite{SpatialComputing12}. The easiest modality is the use of so-called
Location Based Social Networks. The most popular one in 2014 -- Foursquare --
has even been adopted on Mars\footnote{By the \emph{Curiosity} rover:
	\href{http://www.nasa.gov/mission_pages/msl/news/msl20121003.html}%
{http://www.nasa.gov/mission\_pages/msl/news/msl20121003.html}}.\marginpar{Give
more examples, along with references.} Such data enables countless
applications, ranging from personalized recommendation, through urban
planning, to making cities smarter. Besides this actionable side, it also
provides a wealth of insight into human mobility and dynamics.

\medskip

In this work, we devised a similarity measure between urban areas that
fulfills these two objectives. It provide recommendation to travelling users,
as well as comparison different cities and their organization. First we
motivate the need for such a measure through its major use case, before giving
a more precise problem statement.

\medskip

Let us say you have lived in a city for the last years. During this time, you
-- with the potential help of your friends -- have acquired urban knowledge
that you can leverage to find relevant locations where to perform various
tasks. In less awkward words, whether you want to buy milk, listen to loud
music or exhibit your athletic skills, you know where to go and in some cases
why. Yet life is full of surprises and you may be spending a few days in a new
town. Fair enough, but you do not want to renounce to your wide range of
outside activities. This is where our application comes in handy. You select a
location or an area in your familiar city and it returns a place or a zone
sharing similar characteristics in your surrounding. Thus you can boldly
venture into distant shores, reassured that a part of your home will always
travels with you.

\section{Problem definition}
\label{sec:problem}

Let us define formerly this problem. We have a set of cities $\mathcal{C} =
\{c_1, c_2, \ldots, c_n\}$, each of them defined by a enclosing rectangle
$R_{c_i} \subseteq \mathbb{R}^2$. Furthermore, each city $c_i$ contains a
subset $V_{c_i} \subseteq \mathcal{V}$, where $\mathcal{V}$ is a set of
\emph{venues}. A \emph{venue} is an uniquely identified location that can be
visited by individuals. It includes for instance restaurants, shops, airports,
monuments and parks. Finally, each city $c_i$ is described by another subset
$A_{c_i} \subseteq \mathcal{A}$,\marginpar{There is probably a better name
for this (though digital fingerprint is a bit long)} where $\mathcal{A}$ is a
set of \emph{artefacts}. By \emph{artefact}, we refer to a user contributed
discrete point in space and time that can carry an additional payload. Here,
we exploit Flickr photos (with tags) and Foursquare check-ins as type of
artefact\footnote{But it should be possible to extend this definition to other
	kind of data. For instance, a tweet can be seen as point in time and
	space (along with the sentiment it conveys). Or we can consider noise
	and pollution measurements, the number of cars in a given intersection
	during a one minute window, Yelp reviews, etc. Stretching the concept,
	we may even imagine gathering anonymized customer receipts from
	grocery shops.  The payload could then be the proportion in which
	different kind of goods are consumed. Indeed, even for two shops of
	the same company in the same city, it would give information about the
neighboring population.}.

Using this information, we want to devise a similarity measure over the venues
$\mathcal{V}$, namely
 $s: \mathcal{V}^2 \rightarrow \llbracket 0, 1 \rrbracket $. This would
provide a building block for another measure of similarity between two spatial
regions. More precisely, after choosing a home city $c_h$ and a target city
$c_t$\footnote{$t$ can also stand for travel or tourist.}, we want
$s': R_{c_h} \times R_{c_t} \rightarrow \llbracket 0, 1 \rrbracket$.
These two functions answer the following queries:

\begin{enumerate}
\item In the target town, find the venue most similar to a venue $v$ known by
	the user in its home city, that is $\argmax_{v' \in V_{c_t}} s(v,
	v')$.\label{q:point}
\item Instead of a single venue, users may ask for a larger area $r$ matching a
	neighborhood\footnote{or any other kind of subdivision.} from their home
	city, in which case we are looking for $\argmax_{r' \subseteq V_{c_t}}
	s'(r, r')$.\label{q:space}
\end{enumerate}

\section{Contributions}
\marginpar{To be done}

\begin{itemize}
	\item Collect and compute relevant features describing venues.
	\item Seamlessly incorporate heterogeneous \emph{artefacts}.
	\item Deliver interpretable results, which enable principled comparison
		of cities and more user friendly output (it can also potentially
		explain why no sensible result were found for some queries).
		Furthermore, this would address recent concerns about
		Algorithmic Accountability \autocite{Accountability13}. For instance,
		\textcite{Discrimination13} showed that in case of online advertisement
		discrimination, algorithms' opacity makes it difficult to find and
		sanction those responsible.
	\item Efficient search in $\mathbb{R}^2$ instead of brute force for
		the query~\ref{q:space}.
	\item Compare results with existing methods and evaluate them against
		human judging.
\end{itemize}
