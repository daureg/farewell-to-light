\chapter{Introduction}
\label{ch:introduction}

\section{Motivation}

Many people live in cities and nowadays, a lot of them share their
spatio-temporal whereabout on social networks \autocite{SpatialComputing12}.
Such data enables useful applications as well as it provides new insights into
human dynamics. We seize this opportunity to devised a similarity measure
between urban areas that fulfills these two objectives. It gives
recommendation to travelling users, while letting us compare different cities
and their organization. First we motivate the need for such a measure through
its major use case, before giving a more precise problem statement.

\medskip

Let us say you have lived in a city for the last years. During this time, you
have acquired urban knowledge that you can leverage to find relevant locations
where to perform various tasks. In simpler words, whether you want to buy
milk, listen to loud music or exhibit your athletic skills, you know where to
go and in some cases why. Yet life is full of surprises and you may be
spending a few days in a new town. Fair enough, but you do not want to
renounce to your wide range of outside activities. This is where our
application comes in handy. You select a location or an area in your familiar
city and it returns a place or a zone sharing similar characteristics in your
surrounding. Thus you can boldly venture into distant shores, reassured that a
part of your home will always travels with you.

\section{Problem definition}
\label{sec:problem}

Let us define formally this problem. We have a set of cities $\mathcal{C} =
\{c_1, c_2, \ldots, c_n\}$, each of them defined by a enclosing rectangle
$R_{c_i} \subseteq \mathbb{R}^2$. Furthermore, each city $c_i$ contains a
subset $V_{c_i} \subseteq \mathcal{V}$, where $\mathcal{V}$ is a set of
\emph{venues}. A \emph{venue} is an uniquely identified location that can be
visited by individuals. It includes for instance restaurants, shops, airports,
monuments and parks. Finally, each city $c_i$ is described by another subset
$I_{c_i} \subseteq \mathcal{I}$, where $\mathcal{I}$ is a set of \emph{items}.
These \emph{items} are user contributed discrete points in space and time
that can have additional attributes. Here, we exploit Flickr photos (with
tags) and Foursquare check-ins as type of item\footnote{But it should be
	possible to extend this definition to other kind of data. For
	instance, a tweet can be seen as point in time and space (along with
	the sentiment it conveys). Or we can consider noise and pollution
	measurements, the number of cars in a given intersection during a one
	minute window, Yelp reviews, etc. Stretching the concept, we may even
	imagine gathering anonymized customer receipts from grocery shops.
	The payload could then be the proportion in which different kind of
	goods are consumed. Indeed, even for two shops of the same company in
the same city, it would give information about the neighboring population.}.
Items are associated with venues, either directly, like check-ins, or
indirectly, like photos. In the latter case, the relationship between venues
and items is based on spatial distance for instance.

By combining venues own characteristics and their associated items, we want to
devise a similarity measure over the venues $\mathcal{V}$, namely $s:
\mathcal{V}^2 \rightarrow \llbracket 0, 1 \rrbracket $. This provides a
building block for another measure of similarity between two spatial regions.
More precisely, after choosing a home city $c_h$ and a target city
$c_t$\footnote{$t$ can also stand for travel or tourist.}, we want to
construct $s': R_{c_h} \times R_{c_t} \rightarrow \llbracket 0, 1 \rrbracket$.
These two functions answer the following queries:

\begin{enumerate}
\item When users pick a venue $v$ in their home city, find the venue in the
	target town that is the most similar to it. Formally, we compute
	$\argmax_{v' \in V_{c_t}} s(v, v')$.\label{q:point}
\item Instead of a single venue, users ask for a larger area $r$, for instance
	covering a neighborhood\footnote{or any other meaningful subdivision.}
	from their home city. In that case we also return an area:
	$\argmax_{r' \subseteq V_{c_t}} s'(r, r')$.\label{q:space}
\end{enumerate}

\section{Contributions}
\marginpar{To be done}

\begin{itemize}
	\item Collect and compute relevant features describing venues.
	\item Seamlessly incorporate heterogeneous \emph{items}.
	\item Deliver interpretable results. Beyond a single number, the
		similarity, or absence thereof, must be motivated. It enables
		principled comparison of cities and more user friendly output
		(as well as justifying why no sensible result were found for
		some queries). Furthermore, this would address recent
		concerns about Algorithmic Accountability
		\autocite{Accountability13}. For instance,
		\textcite{Discrimination13} showed that in case of online
		advertisement discrimination, opacity of the algorithms makes
		it difficult to find and sanction those responsible.
	\item Efficient search in $\mathbb{R}^2$ instead of brute force for
		the query~\ref{q:space}.
	\item Compare results with existing methods and evaluate them against
		human judging.
\end{itemize}
