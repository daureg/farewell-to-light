\section{Foursquare}

Why people use 4SQ\autocite{FSMotivation11}.

\subsection{Check-ins}

Foursquare check-ins are not available publicly for free, thus one has to
find other way to get them. The easiest one was to reuse an existing dataset.
Between September 2010 and January 2011, \textcite{dataset11} collected
\numprint{22387930} check-ins from various \glspl{lbsn}.

A typical check-in looks like this:
\begin{verbatim}
41418752    19459464870    38.931834    -77.028256
2010-07-25 01:19:02    a95cd7e5e0eddaad
I'm at Meridian Pint (, ) w/ 7 others. http://4sq.com/9GoW57
\end{verbatim}

where we can find the twitter user id, the tweet id, the
latitude and longitude, the local time, a place id and the tweet's text
\footnote{a more thorough description can be found online
at \href{http://infolab.tamu.edu/static/users/zhiyuan/icwsm\_2011\_readme.pdf}%
{\url{http://infolab.tamu.edu/static/users/zhiyuan/icwsm\_2011\_readme.pdf}}.}.

After filtering tweets originating from our twenty cities, we were left with
\numprint{2133752} of them. However, there is no easy way to map twitter place
id to Foursquare venue id\footnote{It even seems that there is not one to one
correspondence between them.}. Therefore, we had to follow the short links
appearing in the tweets to recover those ids. But as the dataset is three
years old, some links are dead. At the end, we found \numprint{1296505}
exploitable check-ins, as showed in \autoref{tab:dataset}.

\medskip

Another way to access check-ins is to use Twitter. Some applications publish
tweets regarding check-in on the behalf of user. Furthermore, Twitter allows
anyone to access a 1\% sample of public tweets\footnote{Using their Public
streams API described here:
\href{https://dev.twitter.com/docs/streaming-apis/streams/public}%
{\url{https://dev.twitter.com/docs/streaming-apis/streams/public}}.}.
\marginpar{refactor script to clean dependencies} Therefore we set up our own
collection process from March to July 2014\footnote{The code can be found on
github: \href{https://github.com/daureg/illalla}%
{\url{https://github.com/daureg/illalla}}.} by selecting tweets with
\url{4sq.com} links, retaining those located in the target cities and that
leads to real check-ins. It adds \numprint{351365} tweets. The format,
described in \autoref{tab:checkinfields}, is rather similar to the previous
one:

\begin{verbatim}
 _id : 533bae24498eea4314f699e1?s=aDw52maNIBKg8_vmChduwSKDm3I
 lid : "4d56bccecff7721ea778b9f5"
 uid : 9037626
 loc : {"type": "Point",
        "coordinates": [2.3979855, 48.85740119]}
 city: "paris"
 time: "2014-04-02T13:28:52Z"
 tid : "451244858809012224"
 tuid: "824282623"
 msg : "Should be forgotten (@ Kingdom of Paradise)
        http://t.co/2CFVAZL1wq"
\end{verbatim}

\begin{table}[ht]
    \centering
    \begin{tabularx}{\textwidth}{lX}
        \toprule
        field & description \\
        \midrule
        \texttt{\_id} & Check-in id and its public signature, which can be used
        to get more information using Foursquare API. \\
        \texttt{lid} & Foursquare venue id \\
        \texttt{uid} & Numerical Foursquare user id \\
        \texttt{loc} & GeoJSON location of the tweet \\
        \texttt{city} & Corresponding city \\
        \texttt{time} & Local time at which the check-in took place (this may differ from the time of the tweet) \\
        \texttt{tid} & Tweet id \\
        \texttt{tuid} & Twitter user id \\
        \texttt{msg} & Textual content of the tweet \\
        \bottomrule
    \end{tabularx}
    \caption[Check-in format]{Description of the fields of the check-in
    object.\label{tab:checkinfields}}
\end{table}

\marginpar{Put those details in Appendix?}
\Textcite{TwitterMongoDB13} describe a similar but more robust setup, that
also use the timeline of selected users.

\begin{table}[ht]
    \centering
    \pgfplotstabletypeset[col sep=&,row sep=\\,fixed, int detect,column type=r,
    alias/old/.initial={2010 check-ins},
    alias/new/.initial={2014 check-ins},
    columns/city/.style={string type,column type=l},
    every head row/.style={before row=\toprule,after row=\midrule},
    every last row/.style={before row=\midrule,after row=\bottomrule}]{%
    city          & 2010 check-ins & 2014 check-ins & photos  & venues \\
    New York      & 408584  & 75635  & 1654289 & 49818  \\
    Los Angeles   & 165463  & 30155  & 743314  & 25110  \\
    Chicago       & 133822  & 28176  & 575306  & 18333  \\
    San Francisco & 104363  & 16241  & 932568  & 12152  \\
    London        & 72674   & 24608  & 1562858 & 16534  \\
    Washington    & 75984   & 17627  & 610263  & 9481   \\
    Seattle       & 51574   & 6744   & 504210  & 7587   \\
    Amsterdam     & 35339   & 4673   & 134459  & 6538   \\
    Houston       & 41037   & 11241  & 32459   & 9080   \\
    Atlanta       & 40798   & 9466   & 215178  & 5766   \\
    Paris         & 32952   & 14296  & 159969  & 11025  \\
    Stockholm     & 10501   & 1768   & 26540   & 3121   \\
    Indianapolis  & 30955   & 5943   & 23740   & 5427   \\
    Moscow        & 17577   & 74134  & 52821   & 21626  \\
    Barcelona     & 21448   & 9273   & 98730   & 7032   \\
    Berlin        & 15098   & 7806   & 193556  & 6430   \\
    St.~Louis     & 17491   & 3415   & 23520   & 2498   \\
    Rome          & 9364    & 5068   & 142113  & 4311   \\
    Prague        & 4757    & 3610   & 41395   & 2943   \\
    Helsinki      & 6724    & 1486   & 21333   & 1956   \\
    total         & 1296505 & 351365 & 7748621 & 226768 \\
}
\caption[Dataset number]{Count of objects in the dataset\label{tab:dataset}}
\end{table}

\subsection{Venues}

Each of these check-ins is associated with a venue. Thus, the second part of
the collection process was to gather information about all of those
\numprint{226768} that appear in at least on check-in. This was done by calling the
appropriate Foursquare API, with result showed in \autoref{tab:venue}.

link to dataset on
\href{http://figshare.com/authors/G\%C3\%A9raud\%20Le\%20Falher/542931}%
{\url{figshare.com}} or \url{http://academictorrents.com}

\subsection{Limitations}

fake checkin\autocite{FakeCheckins12} (although this can be corrected
\autocite{ValidateCheckin13,FindingFake14}

bias in twitter sampling\autocite{TwitterBias14}

situation where people are reluctant to check-in\autocite{Privacy11} (on the
other hands some user are very open: like 5 tweets in less than one hour in a
pub with 3 names of beer through
\url{https://untappd.com/user/fneri3/checkin/85857825?ref=social} and 2 photos
\url{https://foursquare.com/fneri3/checkin/5373fb47498e65814631c3e3?s=D2ximdj1zhKVt_qZy9xpvObiHkc&ref=tw}

fixed categories instead of inferred LDA semantics\autocite{PlaceSemantic14}
