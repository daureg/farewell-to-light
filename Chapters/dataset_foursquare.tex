\section{Foursquare}
\label{sec:foursquare}

Foursquare is the most popular \gls{lbsn} in 2014, claiming more than 50
million users\footnote{See \href{https://foursquare.com/about/}%
{\url{foursquare.com/about}} for more usage statistics.}. Its two main purposes
are: $(i)$ enable users discover new places and $(ii)$ let their friends or the world know
where they are. The first goal is achieved through venues. Venues have their
own web page that displays basic facts (like name, address and type) but also
user contributed information (such as photos, reviews, likes and tips) which
can be used for rating or recommendation. The second goal revolved
around check-in, which represents the visit of a user to a venue at a given
time. It is also associated with a web page, potentially displaying a message
or a photo enclosed with the check-in.

\subsection{Check ins}

Following \fs's privacy policy, check-ins are private.  Yet some users opt to
share their check-ins via \tw\footnote{\href{https://twitter.com/}%
{\url{twitter.com}}} a popular micro-blogging platform. 
As \tw{} allows anyone to access a 1\% sample of public tweets\footnote{Using
the Public streams API described here:
\href{https://dev.twitter.com/docs/streaming-apis/streams/public}%
{\url{dev.twitter.com/docs/streaming-apis/streams/public}}.},
we collected such check-ins
from March to July 2014 by monitoring tweets with
\url{4sq.com} links and retaining those located in the target cities.
Essential informations of these \numprint{1651773} particular tweets are user id, Foursquare venue,
local time, latitude and longitude. Yet they contain other fields, whose
details are given in \autoref{tab:checkinfields}~\pageref{tab:checkinfields}.

In addition, we used a previously released dataset~\autocite{dataset11},
extracting \numprint{1296505} \fs\ check-ins generated between September 2010
and January 2011 in these cities\footnote{a practical description of this
	dataset can be found online at
	\href{http://infolab.tamu.edu/static/users/zhiyuan/icwsm\_2011\_readme.pdf}%
{\url{infolab.tamu.edu/static/users/zhiyuan/icwsm_2011_readme.pdf}}.}.

Finally, we obtained more data by crawling the timeline of users that appear to
post geo located tweets according to the public sample. We apply this approach
in Barcelona, Helsinki, Paris, Rome, San Francisco, Stockholm and Washington,
resulting in \numprint{2087227} additional tweets (even though a significant
portion of them were located in other cities as some users travel).

In total, our \fs\ data consists of more than 5 million check-ins, whose
repartition can be found in \autoref{tab:dataset}.


\begin{table}[ht]
  \small
  \setlength{\tabcolsep}{5pt}
	\centering
	\input{dataset_count.tex}
% City&\multicolumn{3}{c}{Check-ins}&Venues&Photos\\
% 		\cmidrule(r){2-4}
%    & 2010 & 2014 & Timeline &    & \\
% \midrule
% 	\pgfplotstabletypeset[col sep=&,row sep=\\,fixed, int detect,column type=r,
% 	columns/City/.style={string type,column type=l},
% 	outfile=dataset_count.tex,
% 	every head row/.style={before row=\toprule,after row=\midrule},
% 	every last row/.style={before row=\midrule,after row=\bottomrule}]{%
% 		City          & 2010 Check-ins & 2014 Check-ins & Timeline & Venues & Photos \\
% 		New York      & 408584         & 373005         & 362979         & 67803  & 1744890 \\
% 		Los Angeles   & 165463         & 141393         & 119364         & 33829  & 794537  \\
% 		Chicago       & 133822         & 138642         & 71470          & 26249  & 601859  \\
% 		San Francisco & 104363         & 78024          & 296050         & 15641  & 983723  \\
% 		London        & 72674          & 114008         & 89389          & 23618  & 1653489 \\
% 		Washington    & 75984          & 81002          & 291466         & 13242  & 650882  \\
% 		Seattle       & 51574          & 33034          & 35616          & 9685   & 525794  \\
% 		Amsterdam     & 35339          & 24676          & 21647          & 8793   & 148807  \\
% 		Houston       & 41037          & 50642          & 14932          & 13567  & 41827   \\
% 		Atlanta       & 40798          & 43370          & 22797          & 8153   & 228744  \\
% 		Paris         & 32952          & 68297          & 207895         & 17338  & 208895  \\
% 		Stockholm     & 10501          & 8517           & 70377          & 4204   & 37422   \\
% 		Indianapolis  & 30955          & 28006          & 9855           & 7709   & 30908   \\
% 		Moscow        & 17577          & 334472         & 74129          & 48278  & 69826   \\
% 		Barcelona     & 21448          & 45170          & 161122         & 11444  & 130039  \\
% 		Berlin        & 15098          & 40259          & 120151         & 11249  & 226420  \\
% 		St.\ Louis    & 17491          & 15548          & 7044           & 3482   & 33917   \\
% 		Rome          & 9364           & 24409          & 74714          & 7725   & 166537  \\
% 		Prague        & 4757           & 19010          & 8727           & 5748   & 51962   \\
% 		Helsinki      & 6724           & 8394           & 27503          & 3213   & 27849   \\
% 		Total         & 1296505        & 1669878        & 2087227        & 340970 & 8358327 \\
% }
\caption[Dataset number]{Number of check-ins, venues, and photos in each city
Note that \fs\ usage surged in Moscow between 2010 and 2014 whereas the
opposite happened in San Francisco.\label{tab:dataset}}
\end{table}

\subsection{Venues}

As explained, each of these check-ins is associated with a venue. Thus, the second part of
the collection process was to gather information about all of those
\numprint{338669} that appear in at least one check-in. This was done by calling the
appropriate Foursquare API and the obtained object is described in
\autoref{tab:venue}. The main informations are the venue location, its category,
the number of likes, the number of check-ins and the number of unique visitors.
