\section{Foursquare}

As of 2014, Foursquare has become the most popular \gls{lbsn}. But because
its usage it not driven by explicit task solving (like finding direction in a
new city), it sounds to ask why people are using it. According to
\textcite{FSMotivation11}, the main factor are fun (because the application is
gamified and its a lightweight activity to perform when one is bored), social
aspect (see where your friends go and let them know what you are doing) and
discovery (read tips left by previous users and get discount).

\subsection{Check-ins}

Foursquare check-ins are not available publicly for free, thus one has to
find other way to get them. The easiest one was to reuse an existing dataset.
Between September 2010 and January 2011, \textcite{dataset11} collected
\numprint{22387930} check-ins from various \glspl{lbsn}.

A typical check-in looks like this:
\begin{verbatim}
41418752    19459464870    38.931834    -77.028256
2010-07-25 01:19:02    a95cd7e5e0eddaad
I'm at Meridian Pint (, ) w/ 7 others. http://4sq.com/9GoW57
\end{verbatim}

where we can find the twitter user id, the tweet id, the latitude and
longitude, the local time, a place id and the tweet's text\footnote{a more
thorough description can be found online
at \href{http://infolab.tamu.edu/static/users/zhiyuan/icwsm\_2011\_readme.pdf}%
{\url{infolab.tamu.edu/static/users/zhiyuan/icwsm_2011_readme.pdf}}.}.

After filtering tweets originating from our twenty cities, we were left with
\numprint{2133752} of them. However, there is no easy way to map twitter place
id to Foursquare venue id\footnote{Indeed, there is no one to one
correspondence between them.}. Therefore, we had to follow the short links
appearing in the tweets to recover those ids. But as the dataset is three
years old, some links are dead. At the end, we found \numprint{1296505}
exploitable check-ins, as showed in \autoref{tab:dataset}.

\medskip

Another way to access check-ins is to use Twitter. Some applications publish
tweets regarding check-in on the behalf of user. Furthermore, Twitter allows
anyone to access a 1\% sample of public tweets\footnote{Using the Public
streams API described here:
\href{https://dev.twitter.com/docs/streaming-apis/streams/public}%
{\url{dev.twitter.com/docs/streaming-apis/streams/public}}.}.
\marginpar{refactor script to clean dependencies} Therefore we set up our own
collection process from March to July 2014\footnote{The code can be found on
github: \href{https://github.com/daureg/illalla}%
{\url{github.com/daureg/illalla}}.} by selecting tweets with
\url{4sq.com} links, retaining those located in the target cities and that
leads to real check-ins. It adds \numprint{351365} tweets. The format,
described in \autoref{tab:checkinfields}, is rather similar to the previous
one:

\begin{verbatim}
 _id : 533bae24498eea4314f699e1?s=aDw52maNIBKg8_vmChduwSKDm3I
 lid : "4d56bccecff7721ea778b9f5"
 uid : 9037626
 loc : {"type": "Point",
        "coordinates": [2.3979855, 48.85740119]}
 city: "paris"
 time: "2014-04-02T13:28:52Z"
 tid : "451244858809012224"
 tuid: "824282623"
 msg : "Should be forgotten (@ Kingdom of Paradise)
        http://t.co/2CFVAZL1wq"
\end{verbatim}

\begin{table}[ht]
    \centering
    \begin{tabularx}{\textwidth}{lX}
        \toprule
        field & description \\
        \midrule
        \texttt{\_id} & Check-in id and its public signature, which can be used
        to get more information using Foursquare API. \\
        \texttt{lid} & Foursquare venue id \\
        \texttt{uid} & Numerical Foursquare user id \\
        \texttt{loc} & GeoJSON location of the tweet \\
        \texttt{city} & Corresponding city \\
        \texttt{time} & Local time at which the check-in took place (this may differ from the time of the tweet) \\
        \texttt{tid} & Tweet id \\
        \texttt{tuid} & Twitter user id \\
        \texttt{msg} & Textual content of the tweet \\
        \bottomrule
    \end{tabularx}
    \caption[Check-in format]{Description of the fields of the check-in
    object.\label{tab:checkinfields}}
\end{table}

\marginpar{Put those details in Appendix?}
\Textcite{TwitterMongoDB13} describe a similar but more robust setup.
Additionnaly, they obtained more data by crawling the timeline of users that
appear to post geo located tweets according to the public sample. We also
follow this approach in a few cities, resulting in \numprint{300000} more
tweets.

\begin{table}[ht]
    \centering
    \pgfplotstabletypeset[col sep=&,row sep=\\,fixed, int detect,column type=r,
    alias/old/.initial={2010 check-ins},
    alias/new/.initial={2014 check-ins},
    columns/city/.style={string type,column type=l},
    every head row/.style={before row=\toprule,after row=\midrule},
    every last row/.style={before row=\midrule,after row=\bottomrule}]{%
    city          & 2010 check-ins & 2014 check-ins & photos  & venues \\
    New York      & 408584  & 75635  & 1654289 & 49818  \\
    Los Angeles   & 165463  & 30155  & 743314  & 25110  \\
    Chicago       & 133822  & 28176  & 575306  & 18333  \\
    San Francisco & 104363  & 16241  & 932568  & 12152  \\
    London        & 72674   & 24608  & 1562858 & 16534  \\
    Washington    & 75984   & 17627  & 610263  & 9481   \\
    Seattle       & 51574   & 6744   & 504210  & 7587   \\
    Amsterdam     & 35339   & 4673   & 134459  & 6538   \\
    Houston       & 41037   & 11241  & 32459   & 9080   \\
    Atlanta       & 40798   & 9466   & 215178  & 5766   \\
    Paris         & 32952   & 14296  & 159969  & 11025  \\
    Stockholm     & 10501   & 1768   & 26540   & 3121   \\
    Indianapolis  & 30955   & 5943   & 23740   & 5427   \\
    Moscow        & 17577   & 74134  & 52821   & 21626  \\
    Barcelona     & 21448   & 9273   & 98730   & 7032   \\
    Berlin        & 15098   & 7806   & 193556  & 6430   \\
    St.~Louis     & 17491   & 3415   & 23520   & 2498   \\
    Rome          & 9364    & 5068   & 142113  & 4311   \\
    Prague        & 4757    & 3610   & 41395   & 2943   \\
    Helsinki      & 6724    & 1486   & 21333   & 1956   \\
    total         & 1296505 & 351365 & 7748621 & 226768 \\
}
\caption[Dataset number]{Count of objects in the dataset\label{tab:dataset}}
\end{table}

\subsection{Venues}

Each of these check-ins is associated with a venue. Thus, the second part of
the collection process was to gather information about all of those
\numprint{226768} that appear in at least on check-in. This was done by calling the
appropriate Foursquare API, with result showed in \autoref{tab:venue}.

\begin{comments}
link to the dataset on
\href{http://figshare.com/authors/G\%C3\%A9raud\%20Le\%20Falher/542931}%
{\url{figshare.com}} or \url{http://academictorrents.com}
\end{comments}

\subsection{Limitations}

The data collection process described above is easily fooled by dishonest user
claiming to be at some locations even if it is not the case. According to
\autocite{FakeCheckins12}, this can be caused by the desire to \enquote{cheat}
the gaming component of many \glspl{lbsn}, take advantage of monetary
incentives offered by venues owner to go in their establishment or simply
provide allegedly objective alibi to cover for other activity. Solution
involving special hardware and protocol are not within our reach
\autocite{ValidateCheckin13} and even detecting such fake check-ins, by using
temporal tensor decomposition \autocite{FindingFake14} is a problem in itself.
Therefore we did not make any attempt at cleaning the data.

Another concern is the use of Twitter. First because we access only a sample
of all tweets, without any guarantee about its representativeness.
Fortunately, it does not appear that Twitter is purposely trying to trick
researchers \autocite{TwitterBias14}. Second because our tweets span more than
three years, a interval during which Twitter and its user behavior have
changed \autocite{TwitterEvolution14}

The data we collected are largely influenced by privacy consideratiosn.
Even if it would give us a more reliable picture, there are situations in
which people are reluctant (or unable) to check-in \autocite{Privacy11}.
\begin{comments}
On the other hands some users are very open: One posted 5 tweets in less than
one hour in a pub, along with 3 names of beer
\href{https://untappd.com/user/fneri3/checkin/85857825}%
{\url{untappd.com/user/fneri3/checkin/85857825}} and 2 photos
\href{https://foursquare.com/fneri3/checkin/5373fb47498e65814631c3e3?s=D2ximdj1zhKVt_qZy9xpvObiHkc}%
{\url{foursquare.com/fneri3/checkin/5373fb47498e65814631c3e3?s=D2ximdj1zhKVt_qZy9xpvObiHkc}}.
\end{comments}

Finally, through this work, we relied on venues category provided by
Foursquare. Yet it has been showed recently than performing Latent Dirichlet
Allocation on associated text defined finer and more meaningful categorisation
\autocite{PlaceSemantic14}.
