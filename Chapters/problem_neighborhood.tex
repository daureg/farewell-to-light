\subsection*{Setup}
We have two cities, $C_1$ and $C_2$. In each of them lie objects of different
kind (in our case: photos, check-ins and venues). These objects have a
position in $\mathbb{R}^2$, along with other attributes like time, tags,
category or number of visitors.

\subsection*{Goal}

As an input, we are given a polygonal region $R$ contained in $C_1$. The
desired output is the region $R'$ in $C_2$ that is closest to $R$ according
to the metric $\delta_{\theta}$.

\subsection*{Problem}

A region is represented by the $e'$ main objects it contains (the venues).
Each of these venues is in turn described by $d$ numerical features. In
addition, each region have a set of $g$ global features that are not
associated with any specific objects. Thus a region $R$ is fully characterized
by
\begin{align*}
	\phi \colon \mathcal{R} &\to
	\mathbb{R}^{e'\times d} \times \mathbb{R}^g = \mathcal{F} \\
	R &\mapsto (X, x)
\end{align*}

Inside $\mathcal{F}$, two regions can be compared using the
$\theta$-parametrized distance
\begin{align*}
	\delta_{\theta} \colon &\mathcal{F} \times \mathcal{F} \to
	\mathbb{R}^+ \\
	&(X, x) , (X', x') \mapsto d_{R,R'}
\end{align*}

and our our objective is fulfilled by computing, for a given $R \in
\mathcal{R}(C_1)$
\begin{equation}
	\argmin_{R' \in \mathcal{R}(C_2)}\; \delta_{\theta}
	\left(\phi(R), \phi(R')\right)
	\label{e:onetheta}
\end{equation}

More precisely, $\delta_{\theta}$ is defined as follow. Let $X_i$ (respectively
$X'_i$) be the $i$\textsuperscript{th} column of $X$ ($X'$), $P_i$ ($Q_i$) the
associated distribution, $D_{\mathrm{KL}}(P \parallel Q) = \sum_j
P(j)\ln\left(\frac{P(j)}{Q(j)}\right)$ the Kullback--Leibler divergence
between $P$ and $Q$ and $ \mathrm{JSD}(P \parallel Q)=
\frac{1}{2}D_{\mathrm{KL}}(P \parallel M)+\frac{1}{2}D_{\mathrm{KL}}(Q
\parallel M)$ their Jensen--Shannon divergence \autocite{JensenShannon03},
where $M=\frac{1}{2}(P+Q)$. Then
\[
	\delta_{\theta} \left(\phi(R), \phi(R')\right) = \sum_{i=1}^d
	\theta_i \mathrm{JSD}(P_i \parallel Q_i) + \vnorm{x - x'}
\]

\subsection*{Additional considerations}
\subsubsection*{Point Pattern Matching}

A more refined answer would acknowledge that not all regions can be compared in
the same way and thus solve
\begin{equation}
    \argmin_{\theta ,\, R' \in \mathcal{R}(C_2)}\; \delta_{\theta}
    \left(\phi(R), \phi(R')\right) + cost(\theta)
    \label{e:manythetas}
\end{equation}
where the regularisation of $\theta$ ensures that there is still a common
ground in the way similarity is computed.

\eqref{e:manythetas} can be casted as a \emph{Point Pattern
Matching}\footnote{Also known as \emph{Point set registration}.} problem.
Following \textcite{PointPatternMatching08}, let $P$ be the $e'$ points of
$\mathbb{R}^d$ that make up $R$, and let $T$ be the $n$ points of
$\mathbb{R}^d$ lying in $C_2$. The problem asks whether there is a subset $I
\subseteq T$ such that $\delta_{\theta}\left(P, I\right) \leq \epsilon$. In
this case, $\theta$ represents a linear transformation applied to $P$. And
instead of explicit regularisation, it can only be drawn from a fixed class of
transformations. Another constraint is that $I$ must be restricted to subsets
of venues that indeed form neighborhood.

\subsubsection*{The choice of $\delta$}

Here are presented some design alternatives for $\delta$ and why the current
choice was made.

\begin{itemize}
    \item First, weighting venues, either explicitly or, more likely,
        implicitly (for instance by their relative popularity in the region)
        will provide more flexibility. Especially, a fixed fraction of venues
        may have zero weight because we consider they don't really belong to a
        given neighborhood.
    \item There are different distances between two point sets. For instance,
        the bottleneck distance \autocite{Bottleneck96} (the smallest distance
        between pairs of farthest point), the sum of the distance of each
        point to its closest neighbor, or the Hausdorff distance (the largest
        distance between pairs of closest point) and its variation
        \autocite{ModifiedHausdorff94}. From what I have seen, these measures
        have been used mostly in 2 or 3 dimensional space.  Thus it would be
        interesting to see how they behave in higher dimension.  Especially
        since linear transformations in $\mathbb{R}^{31}$ are not very
        intuitive.
    \item Instead of using euclidean distance to compare global feature, some
	    histogram distance \autocite{HistogramDistance02} may be more suited,
        especially to compare the time activity of different regions.
    \item Likewise, moving past the geometric approach, this could also be
        tackled by information theoretic distance over some probability
        distributions, like the Kullback--Leibler divergence\footnote{Or one
        of the many others distances mentionned on Wikipedia:
        \href{https://en.wikipedia.org/wiki/Statistical_distance}%
        {\url{en.wikipedia.org/wiki/Statistical_distance}}}. Or as suggested
        by the paper from Arto \autocite{InfoMatching10}, by looking at the
        mutual information between two regions.
    \item Finally, we could build a graph out of venues. It may for instance
        contains information about their relation in the original 2D space. Or
        be tailored to take categories into account \autocite{fitting09}. Then
        they can be compared using appropriate similarity
        \autocite{GraphSimilarity14}. The drawback is that it seems somewhat
        artificial and contrived.
\end{itemize}

Because the venues dimensions do not share a common semantic (as it is the case
of the $x$ and $y$ axis of a plane), it feels rather awkward to use a geometric
approach, whereas the probability distances operate over one dimension at a
time.

Yet, these three main classes of distance -- and their numerous variations --
all measure distance between two regions. Therefore, we can consider these
distances as scores given to a candidate region by different judges and try to
combine them by weighted vote.

One disadvantage of probabilistic distances is that since they need to
estimate density, they would probably not perform well when there are not
enough venues to get reliable estimation.  We also need to choose between
simple histogram counting or more expensive kernel density estimation.
