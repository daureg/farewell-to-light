%********************************************************************
% Appendix
%*******************************************************
% If problems with the headers: get headings in appendix etc. right
%\markboth{\spacedlowsmallcaps{Appendix}}{\spacedlowsmallcaps{Appendix}}
\clearpage
\newgeometry{hmargin=2.1cm,vmargin=0.7cm}
\chapter{Dataset format}
\label{chap:appendix}

\begin{table}[ht]
    \centering
    \begin{tabularx}{\textwidth}{lXX}
        \toprule
        field & description & example\\
        \midrule
        \texttt{\_id} & Check-in id and its public signature, which can be
used to get more information using Foursquare API. &
533bae24498eea4314f699e1?s=aDw52maNIBKg8\_vmChduwSKDm3I \\
        \texttt{lid} & Foursquare venue id & "4d56bccecff7721ea778b9f5"\\
        \texttt{uid} & Numerical Foursquare user id & 9037626\\
        \texttt{loc} & GeoJSON location of the tweet & {"type": "Point", "coordinates": [2.3979855, 48.85740119]}\\
        \texttt{city} & Corresponding city & "paris"\\
        \texttt{time} & Local time at which the check-in took place (this may
differ from the time of the tweet) & "2014-04-02T13:28:52Z"\\
        \texttt{tid} & Tweet id & "451244858809012224"\\
        \texttt{tuid} & Twitter user id & "824282623"\\
        \texttt{msg} & Textual content of the tweet & "Should be forgotten (@
Kingdom of Paradise) \url{http://t.co/2CFVAZL1wq}"\\
        \bottomrule
    \end{tabularx}
    \caption[Check-in format]{Description of the fields of the check-in
    object.\label{tab:checkinfields}}
\end{table}

\begin{table}[hb]
	\centering
	\begin{tabularx}{\textwidth}{lXX}
		\toprule
		field & description & example \\
		\midrule
		\texttt{\_id}   & Flickr photo id                                    & 12683984675 \\
		\texttt{loc}    & GeoJSON location of the photo                      & {"type": "Point", "coordinates": [25.009667, 60.227586]} \\
		\texttt{uid}    & Flickr user id                                     & "40747751@N08" \\
		\texttt{taken}  & Local time of the shooting according to the camera & ISODate("2014-02-21T18:40:21Z") \\
		\texttt{title}  & User given title                                   & "One Way to Lock a Bicycle" \\
		\texttt{hint}   & city name                                          & "helsinki" \\
		\texttt{tags}   & Normalized user given tags                         & ["bicycle", "lock", "chain", padlock, "locked"] \\
		\texttt{venue}  & Foursquare venue id                                & null \\
		\texttt{upload} & UTC time at which the photo was uploaded           & ISODate("2014-02-22T00:28:32Z") \\
		\texttt{farm}   & Used to obtain a link to the picture file          & 6 \\
		\texttt{server} & Used to obtain a link to the picture file          & "5506" \\
		\texttt{secret} & Used to obtain a link to the picture file          & "74fbd5b60a" \\
		\bottomrule
	\end{tabularx}
	\caption[Photo format]{Description of the fields of the photo
	object.\label{tab:photo}}
\end{table}

\begin{table}[ht]
    \centering
    \begin{tabularx}{\textwidth}{lXX}
        \toprule
        field & description & example \\
        \midrule
\texttt{\_id}          & Foursquare venue id                             & "4ba22d4cf964a5207ce137e3" \\
\texttt{name}          & Name of the venue                               & "Ravintola Oiva" \\
\texttt{loc}           & GeoJSON location according to Foursquare        & {"type" : "Point", "coordinates": [24.951011, 60.181716084480286]} \\
\texttt{cat}           & id of the primary category                      & "4bf58dd8d48988d11b941735" \\
\texttt{cats}          & Potential additional categories                 & ["4bf58dd8d48988d1c4941735", "4bf58dd8d48988d143941735"] \\
\texttt{checkinsCount} & Total number of check-ins in Foursquare         & \numprint{2556} \\
\texttt{usersCount}    & Number of unique user in Foursquare             & \numprint{1321} \\
\texttt{tipCount}      & Number of tip left by users                     & 13 \\
\texttt{price}         & Price range, from 1 to 4                        & 2 \\
\texttt{rating}        & Rating over 10                                  & 7.82 \\
\texttt{createdAt}     & Time at which the venue was added to Foursquare & ISODate("2010-03-18T15:40:28Z") \\
\texttt{mayor}         & User id of the mayor at the time of collection  & 18123276 \\
\texttt{tags}          & User contributed tags                           & ["karaoke"] \\
\texttt{shortUrl}      & Short URL appearing in tweets                   & "\url{http://4sq.com/bV8IRQ}" \\
\texttt{canonicalUrl}  & Full URL                                        & "\url{https://foursquare.com/v/ravintola-oiva/4ba22d4cf964a5207ce137e3}" \\
\texttt{likes}         & Total number of like in Foursquare              & 3 \\
\texttt{likers}        & List of at most ten users liking this venue     & [14695530, 9936722, 37114401] \\
\texttt{city}          & City name                                       & "helsinki" \\
\texttt{closed}        & True when Foursquare says the venue is closed   & null \\
\texttt{hours}         & Always \texttt{null}                            & null \\
        \bottomrule
    \end{tabularx}
    \caption[Venue format]{Description of the fields of the venue
    object.\label{tab:venue}}
\end{table}
\restoregeometry

\chapter{More acknowledgments}
\sec{chap:ack}

All this work was done on Linux workstations, using many other open source tool
such as \LaTeX, the Python programming language, the IPython environment
\autocite{IPython07} and many scientific libraries:
\begin{itemize}
	\item Numpy \autocite{Numpy11}
	\item Scipy \autocite{Scipy14}
	\item Matplotlib \autocite{Matplotlib07}
	\item Scikit-learn \autocite{Scikit12}
	\item Pandas \autocite{Pandas10}
\end{itemize}

The calculations were performed using computer resources within the Aalto
University School of Science "Science-IT" project.
