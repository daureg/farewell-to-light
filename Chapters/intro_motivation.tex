\chapter{Introduction}
\label{ch:introduction}

\section{Motivation}

More and more people are living in cities\footnote{According to an
\href{http://esa.un.org/unup/pdf/WUP2011_Highlights.pdf}{UN report},
\enquote{the population living in urban areas is projected to [pass] from 3.6
billion in 2011 to 6.3 billion in 2050.}}. This simple observation leads to
two questions: how to help urban dwellers make informed everyday decisions and
how to understand these large and complex systems. Fortunately, now is good
time to tackle these issues. Indeed, thanks to social networks and mobile
devices, we know where and when people are active within
cities\autocite{SpatialComputing12}. We seize this opportunity to devised a
similarity measure between urban areas. First we motivate the need for such a
measure through a typical use case:

\begin{quote}
Let us say you have lived in a city for the last years. During this time,
you have acquired urban knowledge that you can leverage to find relevant
locations where to perform various tasks. In simpler words, whether you want
to buy milk, listen to loud music or exhibit your athletic skills, you know
where to go. Yet life is full of surprises and you may be spending a few
days in a new town. Fair enough, but you do not want to renounce to your
wide range of outside activities. This is where our application comes in
handy. You select a location or an area in your familiar city and it shows
you a place or a zone sharing similar characteristics in your surrounding.
Thus you can boldly venture into distant shores, reassured that a part of
your home will always travels with you.
\end{quote}

This short illustration gives an intuivive definition of the problem we are
trying to solve. Namely, an user picks a small part of a city and we represent
it by a bag of all venues it contains. These venues are associated with some
features and we want to find, efficiently, a set of venues that shares similar
characteristics. Yet for this result to be meaningful, we add the constraint
that the venues must delineate a neighborhood, that is a small and compact
subset of the whole city.

In the next section, we will give a more precise problem statement. Before
that, notice that although we define it in terms of cities and venues, it can
be generalized to other settings. All we need is a set of disconnected metric
spaces\footnote{Here two dimensional cities.}, points lying in those spaces
and described by some features\footnote{Here the venues described by their
activity.} and a set of latent high-level characteristics associated with set
of points\footnote{For instance here, \enquote{nightlife neighborhood} or
\enquote{touristic hotspot}.}. Here are some examples that can be modeled
under this framework:
\begin{itemize}
	\item The social graph of students in several universities. The
		metric is a distance in the graph and featured point are
		students. Picking some of them, we posit that they are close because
        they share the same degree program or some other demographic. We
        want to know if the same group exists elsewhere.
	\item Graph of twitter users in different cities or countries. They are
        described by the topic distribution of their tweets. We ask whether
        some community of interest exist in several places in the world.
    \item The set of all books ever written, grouped by centuries and
        characterised by their style and main themes. Choosing all pieces
        written during the Enlightenment, it would tell if there was a
        similar movement in the twentieth century.
\end{itemize}
Even though these examples are not all relevant, they provide additionnal
motivation to solve the problem at hand.
