\chapter{Introduction}
\label{ch:introduction}

\section{Motivation}

Many people live in cities and nowadays, a lot of them share their
spatio-temporal whereabout on social networks \autocite{SpatialComputing12}.
Such data enables useful applications as well as it provides new insights into
human dynamics. We seize this opportunity to devised a similarity measure
between urban areas that fulfills these two objectives. It gives
recommendation to travelling users, while letting us compare different cities
and their organization. First we motivate the need for such a measure through
its major use case, before giving a more precise problem statement.

\medskip

Let us say you have lived in a city for the last years. During this time, you
have acquired urban knowledge that you can leverage to find relevant locations
where to perform various tasks. In simpler words, whether you want to buy
milk, listen to loud music or exhibit your athletic skills, you know where to
go and in some cases why. Yet life is full of surprises and you may be
spending a few days in a new town. Fair enough, but you do not want to
renounce to your wide range of outside activities. This is where our
application comes in handy. You select a location or an area in your familiar
city and it returns a place or a zone sharing similar characteristics in your
surrounding. Thus you can boldly venture into distant shores, reassured that a
part of your home will always travels with you.
