\section{Motivation}
\label{sec:motivation}

More and more people are living in cities\footnote{According to an
\href{http://esa.un.org/unup/pdf/WUP2011_Highlights.pdf}{UN report}
\autocite{UNreport12}, \enquote{the population living in urban areas is
projected to [pass] from 3.6 billion in 2011 to 6.3 billion in 2050.}}. This
simple observation leads to two questions: how to help urban dwellers make
informed everyday decisions and how to understand these large and complex
systems. Fortunately, now is a good time to tackle these issues. Indeed, thanks
to social networks and mobile devices, we know where and when people are
active within cities \autocite{SpatialComputing12}. We seize this opportunity
to devise a similarity measure between urban areas. First we motivate the
need for such a measure through a typical use case:

\begin{quote}
Let us say you have lived in a city for the last years. During this time,
you have acquired urban knowledge that you can leverage to find relevant
locations where to perform various tasks. In simpler words, whether you want
to buy milk, listen to loud music or exhibit your athletic skills, you know
where to go. Yet life is full of surprises and you may be spending a few
days in a new town. Fair enough, but you do not want to renounce to your
wide range of outside activities. This is where our application comes in
handy. You select a location or an area in your familiar city and it shows
you a place or a zone sharing similar characteristics in your surroundings.
Thus you can boldly venture into distant shores, reassured that a part of
your home will always travel with you.
\end{quote}

This short illustration gives an intuitive definition of the problem we are
trying to solve. Namely, a user picks a small part of a city and we represent
it by a bag of all venues it contains. These venues are associated with some
features and we want to find, efficiently, a set of venues that shares similar
characteristics. Yet for this result to be meaningful, we add the constraint
that the venues must delineate a neighborhood, that is a small and compact
subset of the whole city.

As showed, this problem has applications to recommending locations in a city.
But it is also applicable in the analysis of cities and urban planning. For
instance, when applied to the neighborhoods of one city, our techniques allow
to identify neighborhoods that are similar to each other, and thus can help us
understand what is going on in each area, what are the hubs of different
activities, how citizens are experiencing the city, and how they are utilizing
its resources. 
