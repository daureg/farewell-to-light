\chapter{Dataset}
\label{chap:dataset}

There are many ways to obtain information about places. The most
straightforward is to ask directly people. For instance, \textcite{Curated14}
\marginpar{describe better} asked residents of Pittsburgh to build a city
guide by sharing their opinion about locations they know well.
Mine text, like yelp review \autocite{YelpReview14}
Image analysis to describe place atmosphere
Transportation data to see which place are connected

In this work, we choose to focus on trace of human activity. Namely we
collected Foursquare check-ins and Flickr photos' metadata. The first step of
this process was to choose a fixed number of cities.

According to \href{http://www.4sqstat.com/}{4sqstat.com}, we first selected
the ten cities in the US with the highest number of check-ins: New York,
Washington, San Francisco, Atlanta, Indianapolis, Los Angeles, Seattle,
Houston, St. Louis and Chicago. We did something similar in Europe,
resulting in the selection of London, Paris, Berlin, Rome, Prague, Moscow,
Amsterdam, Helsinki, Stockholm and Barcelona. This is admittedly a choice
biased by our own history and the ease to gather user input. Indeed, Rio
de Janeiro or Tokyo are more active than Helsinki or Stockholm, but less
familiar to us and to many students of Aalto University, making results more
difficult to interpret.

\section{Foursquare}

Why people use 4SQ\autocite{FSMotivation11}

\subsection{Check-ins}

Foursquare check-ins are not available publicly for free, thus one has to
find other way to get them. The easiest one was to reuse an existing dataset.
Between September 2010 and January 2011, \textcite{dataset11} collected
\numprint{22387930} check-ins  from various LBSN.

A typical check-in looks like this:
\begin{verbatim}
41418752    19459464870    38.931834    -77.028256
2010-07-25 01:19:02    a95cd7e5e0eddaad
I'm at Meridian Pint (, ) w/ 7 others. http://4sq.com/9GoW57
\end{verbatim}

where we can find the twitter user id, the tweet id, the
latitude and longitude, the local time, a place id and the tweet's text
\footnote{a more thorough description can be found online
at \href{http://infolab.tamu.edu/static/users/zhiyuan/icwsm\_2011\_readme.pdf}%
{\url{http://infolab.tamu.edu/static/users/zhiyuan/icwsm\_2011\_readme.pdf}}.}.

After filtering tweets originating from our twenty cities, we were left with
\numprint{2133752} of them. However, there is no easy way to map twitter place
id to Foursquare venue id\footnote{It even seems that there is not one to one
correspondence between them.}. Therefore, we had to follow the short links
appearing in the tweets to recover those ids. But as the dataset is rather old,
some links are dead. At the end, we found \numprint{1296505} exploitable
check-ins, as showed in \autoref{tab:dataset}.

\medskip

Another way to access check-ins is to use Twitter. Some applications publish
tweets regarding check-in on the behalf of user. Furthermore, Twitter allows
anyone to access a small sample of public tweets\footnote{Using their Public
    streams API described here:
    \href{https://dev.twitter.com/docs/streaming-apis/streams/public}%
{\url{https://dev.twitter.com/docs/streaming-apis/streams/public}}.}.
\marginpar{refactor script to clean dependencies} Therefore we set up our own
collection process from March to July 2014\footnote{The code can be found on
    github: \href{https://github.com/daureg/illalla}%
{\url{https://github.com/daureg/illalla}}.} by selecting tweets with
\url{4sq.com} links, retaining those located in the target cities and that
leads to real check-ins. It adds \numprint{351365} tweets. The format,
described in \autoref{tab:checkinfields}, is rather similar to the previous
one:

\begin{verbatim}
 _id : 533bae24498eea4314f699e1?s=aDw52maNIBKg8_vmChduwSKDm3I
 lid : "4d56bccecff7721ea778b9f5"
 uid : 9037626
 loc : {"type": "Point",
        "coordinates": [2.3979855, 48.85740119]}
 city: "paris"
 time: "2014-04-02T13:28:52Z"
 tid : "451244858809012224"
 tuid: "824282623"
 msg : "Should be forgotten (@ Kingdom of Paradise)
        http://t.co/2CFVAZL1wq"
\end{verbatim}

\begin{table}[ht]
    \centering
    \begin{tabularx}{\textwidth}{lX}
        \toprule
        field & description \\
        \midrule
        \texttt{\_id} & Check-in id and its public signature, which can be used
        to get more information using Foursquare API. \\
        \texttt{lid} & Foursquare venue id \\
        \texttt{uid} & Numerical Foursquare user id \\
        \texttt{loc} & GeoJSON location of the tweet \\
        \texttt{city} & Corresponding city \\
        \texttt{time} & Local time at which the check-in took place (this may differ from the time of the tweet) \\
        \texttt{tid} & Tweet id \\
        \texttt{tuid} & Twitter user id \\
        \texttt{msg} & Textual content of the tweet \\
        \bottomrule
    \end{tabularx}
    \caption[Check-in format]{Description of the fields of the check-in
    object.\label{tab:checkinfields}}
\end{table}

\begin{table}[ht]
    \centering
    \pgfplotstabletypeset[col sep=&,row sep=\\,fixed, int detect,column type=r,
    alias/old/.initial={2010 check-ins},
    alias/new/.initial={2014 check-ins},
    columns/city/.style={string type,column type=l},
    every head row/.style={before row=\toprule,after row=\midrule},
    every last row/.style={before row=\midrule,after row=\bottomrule}]{%
    city          & 2010 check-ins & 2014 check-ins & photos  & venues \\
    New York      & 408584  & 75635  & 1654289 & 49818  \\
    Los Angeles   & 165463  & 30155  & 743314  & 25110  \\
    Chicago       & 133822  & 28176  & 575306  & 18333  \\
    San Francisco & 104363  & 16241  & 932568  & 12152  \\
    London        & 72674   & 24608  & 1562858 & 16534  \\
    Washington    & 75984   & 17627  & 610263  & 9481   \\
    Seattle       & 51574   & 6744   & 504210  & 7587   \\
    Amsterdam     & 35339   & 4673   & 134459  & 6538   \\
    Houston       & 41037   & 11241  & 32459   & 9080   \\
    Atlanta       & 40798   & 9466   & 215178  & 5766   \\
    Paris         & 32952   & 14296  & 159969  & 11025  \\
    Stockholm     & 10501   & 1768   & 26540   & 3121   \\
    Indianapolis  & 30955   & 5943   & 23740   & 5427   \\
    Moscow        & 17577   & 74134  & 52821   & 21626  \\
    Barcelona     & 21448   & 9273   & 98730   & 7032   \\
    Berlin        & 15098   & 7806   & 193556  & 6430   \\
    St.~Louis     & 17491   & 3415   & 23520   & 2498   \\
    Rome          & 9364    & 5068   & 142113  & 4311   \\
    Prague        & 4757    & 3610   & 41395   & 2943   \\
    Helsinki      & 6724    & 1486   & 21333   & 1956   \\
    total         & 1296505 & 351365 & 7748621 & 226768 \\
}
\caption[Dataset number]{Count of objects in the dataset\label{tab:dataset}}
\end{table}

\subsection{Venues}

Each of these check-ins is associated with a venue. Thus, the second part of
the collection process was to gather information about all of those
\numprint{226768} that appear in at least on check-in. This was done by calling the
appropriate Foursquare API, with result showed in \autoref{tab:venue}.

link to dataset on
\href{http://figshare.com/authors/G\%C3\%A9raud\%20Le\%20Falher/542931}%
{\url{figshare.com}} or \url{http://academictorrents.com}

\subsection{Limitations}

fake checkin\autocite{FakeCheckins12} (although this can be corrected
\autocite{ValidateCheckin13,FindingFake14}

bias in twitter\autocite{TwitterBias14}

situation where people are reluctant to check-in\autocite{Privacy11}

fixed categories instead of inferred LDA semantics\autocite{PlaceSemantic14}

\section{Flickr}

Using the Flickr API, we downloaded metadata\footnote{But no image at all.}
from every photo satisfying a set of criteria: they contained at least one
tag, they were located inside one of the chosen city and they have been
uploaded after January 1\textsuperscript{st}, 2008. This yields \numprint{7748621}
photos but as showed in \autoref{tab:dataset}, there are a factor of 77
between figures in New-York and Helsinki.

More precisely, in addition to tags and location, we know when each photos was
taken and uploaded, by which user and what title was given to it (the title
was not used except when it contained hashtags, which were converted to tags).
Thus a typical data point looks like what is described in \autoref{tab:photo}.

\subsection{Limitations}

This retrieval process was naturally not perfect. In addition to some API
calls returning strange results, the casual nature of the data explains their
inherent noise.

\begin{itemize}
	\item While timestamp issued by mobile phones are likely to be correct,
 as their internal clock is synchronized by internet, this may not
 always be the case for dedicated cameras. More concerning than usual
 drift of low quality clock is the situation of tourists coming from
 different timezone. Yet as I could not think of any simple solution to
 that problem, I just ignored it and carried on.
	\item To ensure the quality of the localization, I restricted myself to
 photos whose precision is deemed \enquote{street level} by Flickr. The
 potential problem is that it would cost an extra request to know
 whether this location was given by GPS (in which case the camera
 position is accurate) or by the user at upload time. In the latter
 case, in addition to the general imprecision of the method, it is
 ambiguous whether this location refer the place where the photo was
 shot or the position of the photo's subject\footnote{Think of a bridge
 taken from a nearby hill.}.
	\item Finally, without additional request, the tags obtained are those
 normalized by Flickr. This normalization is not bijective but it is
 assumed that two tags with the same normalized form were close in the
 first place.
\end{itemize}

\section{Human input}

unsupervised, can't really use history as traditionnal recommender because of
sparsity (even tough, still need to match restaurants visited in two cities).
Brief description of the website with a pictures, list of questions and
categories, number of answers, analysis of the results(?).

\section{Exploration}

\subsection{Time}
Besides their own issues, these two datasets suffers from
biased user distribution\autocite{Weird10}, heavy tail (half of twitter users
have never tweet
\href{http://twopcharts.com/twitteractivitymonitor}{\url{twopcharts.com/twitteractivitymonitor}}

\section{Chosen representation}
Based on the insights gained by exploring the dataset, we settle to represents
each venues by the numerical vector presented in \autoref{tab:venuefeatures}.

\clearpage

\begin{table}[p]
    \centering
    \begin{tabularx}{\textwidth}{lX}
        \toprule
        index & description \\
        \midrule
	\datasetRow{}0 & Number of likes \\
	\datasetRow{}1 & Number of unique users \\
	\datasetRow{}2 & Number of total check-ins \\
        3 & User entropy \\
        4 & Venue density \\
	\datasetRow{}5 & Venue top level category \\
        6 & \enquote{Arts \& Entertainment} venues around \\
        7 & \enquote{College \& University} venues around \\
        8 & \enquote{Food} venues around \\
        9 & \enquote{Nightlife Spot} venues around \\
        10 & \enquote{Outdoors \& Recreation} venues around \\
        11 & \enquote{Shop \& Service} venues around \\
        12 & \enquote{Professional \& Other Places} venues around \\
        13 & \enquote{Residence} venues around \\
        14 & \enquote{Travel \& Transport} venues around \\
	15 & Ratio of photos over check-ins \\
	16 & Ratio of photos associated to the venue over photos linked to other venues \\
	17 & More than half of the check-ins occurs during the week-end \\
	18 & Frequency of check-ins between 2 a.m. and 6 a.m. \\
	19 & Frequency of check-ins between 6 a.m. and 10 a.m. \\
	20 & Frequency of check-ins between 10 a.m. and 2 p.m. \\
	21 & Frequency of check-ins between 2 p.m. and 6 p.m. \\
	22 & Frequency of check-ins between 6 p.m. and 10 p.m. \\
	23 & Frequency of check-ins between 10 p.m. and 2 a.m. \\
        \bottomrule
    \end{tabularx}
    \caption[Venue features]{Feature representation of Foursquare venues,
	    taking surrounding into account. \colorbox{Honeydew}{Shaded
	    features} are provided directly by Foursquare database where the
	    others were computed solely based on the partial information
	    contained in the dataset \label{tab:venuefeatures}}
\end{table}

\newgeometry{hmargin=0.5cm,vmargin=0.6cm}
\begin{table}[t]
	\centering
	\begin{tabularx}{\textwidth}{lXX}
		\toprule
		field & description & example \\
		\midrule
		\texttt{\_id}   & Flickr photo id                                    & 12683984675 \\
		\texttt{loc}    & GeoJSON location of the photo                      & {"type": "Point", "coordinates": [25.009667, 60.227586]} \\
		\texttt{uid}    & Flickr user id                                     & "40747751@N08" \\
		\texttt{taken}  & Local time of the shooting according to the camera & ISODate("2014-02-21T18:40:21Z") \\
		\texttt{title}  & User given title                                   & "One Way to Lock a Bicycle" \\
		\texttt{hint}   & city name                                          & "helsinki" \\
		\texttt{tags}   & Normalized user given tags                         & ["bicycle", "lock", "chain", padlock, "locked"] \\
		\texttt{venue}  & Foursquare venue id                                & null \\
		\texttt{upload} & UTC time at which the photo was uploaded           & ISODate("2014-02-22T00:28:32Z") \\
		\texttt{farm}   & Used to obtain a link to the picture file          & 6 \\
		\texttt{server} & Used to obtain a link to the picture file          & "5506" \\
		\texttt{secret} & Used to obtain a link to the picture file          & "74fbd5b60a" \\
		\bottomrule
	\end{tabularx}
	\caption[Photo format]{Description of the fields of the photo
	object.\label{tab:photo}}
\end{table}
\begin{table}[b]
    \centering
    \begin{tabularx}{\textwidth}{lXX}
        \toprule
        field & description & example \\
        \midrule
\texttt{\_id}          & Foursquare venue id                             & "4ba22d4cf964a5207ce137e3" \\
\texttt{name}          & Name of the venue                               & "Ravintola Oiva" \\
\texttt{loc}           & GeoJSON location according to Foursquare        & {"type" : "Point", "coordinates": [24.951011, 60.181716084480286]} \\
\texttt{cat}           & id of the primary category                      & "4bf58dd8d48988d11b941735" \\
\texttt{cats}          & Potential additional categories                 & ["4bf58dd8d48988d1c4941735", "4bf58dd8d48988d143941735"] \\
\texttt{checkinsCount} & Total number of check-ins in Foursquare         & \numprint{2556} \\
\texttt{usersCount}    & Number of unique user in Foursquare             & \numprint{1321} \\
\texttt{tipCount}      & Number of tip left by users                     & 13 \\
\texttt{price}         & Price range, from 1 to 4                        & 2 \\
\texttt{rating}        & Rating over 10                                  & 7.82 \\
\texttt{createdAt}     & Time at which the venue was added to Foursquare & ISODate("2010-03-18T15:40:28Z") \\
\texttt{mayor}         & User id of the mayor at the time of collection  & 18123276 \\
\texttt{tags}          & User contributed tags                           & ["karaoke"] \\
\texttt{shortUrl}      & Short URL appearing in tweets                   & "\url{http://4sq.com/bV8IRQ}" \\
\texttt{canonicalUrl}  & Full URL                                        & "\url{https://foursquare.com/v/ravintola-oiva/4ba22d4cf964a5207ce137e3}" \\
\texttt{likes}         & Total number of like in Foursquare              & 3 \\
\texttt{likers}        & List of at most ten users liking this venue     & [14695530, 9936722, 37114401] \\
\texttt{city}          & City name                                       & "helsinki" \\
\texttt{closed}        & True when Foursquare says the venue is closed   & null \\
\texttt{hours}         & Always \texttt{null}                            & null \\
        \bottomrule
    \end{tabularx}
    \caption[Venue format]{Description of the fields of the venue
    object.\label{tab:venue}}
\end{table}
\restoregeometry
