\section{Exploration}

While collecting this data, it was helpful to explore them, in order to
discover feature that could characterize places and help cluster them into
similar group.

\subsection{Time}

The time at which check-ins occur is undoubtedly a important variable. For
instance, \textcite{UrbanStory12} performed topic modelling from the text of
check-in tweets and showed that various topics exhibit daily or weekly
patterns. Likewise, \textcite{TimeCluster13} clustered cells of a rectangular
grid according to their repartition of check-ins over categories and time of
day.

Thus we first looked at when check-ins are performed, and how it differed from
city to city \autoref{fig:time_checkin}. Then we adopted a venue-centric view
and clustered locations according to which time they are active
\autoref{fig:day_cluster}, which also depend of the city
\autoref{fig:time_cluster_size}.

\begin{figure}[hbt]
    \begin{subfigure}[b]{\textwidth}
    \centering
    \iftoggle{EXTERNALPGF}{%
        \includegraphics[width=\textwidth]{daily_checkin}
    }{%
        \begin{tikzpicture}
	\begin{axis}[xmin=0, xmax=24,
            unit vector ratio*=1 1.2,
	    axis y line=none,
            ymin=0, ymax=8.8,
            % title={Hourly check-ins frequency through the day},
            % xlabel={hour},
            % ylabel={percentage},
            legend style={at={(0.0,0.8)},anchor=west,draw=none,fill=none},
            grid=major
		]
\addplot[DodgerBlue,thin, smooth]table [x={time}, y={freq}] {newyork_day.dat};
\addlegendentry{New-York}
\addplot[FireBrick,thin, smooth]table [x={time}, y={freq}] {london_day.dat};
\addlegendentry{London}
\addplot[LimeGreen,thin, smooth]table [x={time}, y={freq}] {barcelona_day.dat};
\addlegendentry{Barcelona}
\addplot[Orange,thin, smooth]table [x={time}, y={freq}] {helsinki_day.dat};
\addlegendentry{Helsinki}
\addplot[LightGray,thin, smooth]table [x={time}, y={freq}] {global_day.dat};
\addlegendentry{Total}
\end{axis}
\end{tikzpicture}

    }
    \caption[Pattern of check-in during the day]{Generally, the activity is at
    its lowest around 5 a.m.\ and during the day, there are three peaks: one
when people go to work in the morning, one in the middle of the day and the
last one at the end of the evening. Yet, depending of the city, these peaks do
not happen at the same time, nor with the same intensity.
\label{fig:daily_checkin}}
    \end{subfigure}

    \begin{subfigure}[b]{\textwidth}
    \centering
    \iftoggle{EXTERNALPGF}{%
        \includegraphics[width=\textwidth]{weekly_checkin}
    }{%
        \begin{tikzpicture}
    \begin{axis}[x={11},
            % title={Check-ins frequency through the week},
            % xlabel={time},
            % ylabel={percentage},
		    ymax=3.4,axis y line=none,
            legend
	    style={at={(0.76,0.05)},anchor=south,draw=none,font=\normalsize},
            grid=major,
            xticklabels={Mon 0--4, Mon 4--8, Mon 8--12, Mon 12--16, Mon
                16--20, Mon 20--24, Tue 0--4, Tue 4--8, Tue 8--12, Tue 12--16,
                Tue 16--20, Tue 20--24, Wed 0--4, Wed 4--8, Wed 8--12, Wed
                12--16, Wed 16--20, Wed 20--24, Thu 0--4, Thu 4--8, Thu 8--12,
                Thu 12--16, Thu 16--20, Thu 20--24, Fri 0--4, Fri 4--8, Fri
                8--12, Fri 12--16, Fri 16--20, Fri 20--24, Sat 0--4, Sat 4--8,
                Sat 8--12, Sat 12--16, Sat 16--20, Sat 20--24, Sun 0--4, Sun
            4--8, Sun 8--12, Sun 12--16, Sun 16--20, Sun 20--24},
            xtick={0,...,41},
            x tick label style={rotate=45,anchor=east}
        ]

\addplot[DodgerBlue, smooth]table [x={time}, y={freq}] {newyork_week.dat};
\addlegendentry{New-York}
\addplot[FireBrick, smooth]table [x={time}, y={freq}] {london_week.dat};
\addlegendentry{London}
\addplot[LimeGreen, smooth]table [x={time}, y={freq}] {barcelona_week.dat};
\addlegendentry{Barcelona}
\addplot[Orange, smooth]table [x={time}, y={freq}] {helsinki_week.dat};
\addlegendentry{Helsinki}
\addplot[LightGray, smooth]table [x={time}, y={freq}] {global_week.dat};
\addlegendentry{Total}
% \addplot[DodgerBlue, smooth]table [x={time}, y={freq}] {helsinki_week.dat};
% \addlegendentry{Helsinki}
% \addplot[FireBrick, smooth]table [x={time}, y={freq}] {paris_week.dat};
% \addlegendentry{Paris}
% \addplot[LimeGreen, smooth]table [x={time}, y={freq}] {barcelona_week.dat};
% \addlegendentry{Barcelona}
% \addplot[Black, smooth]table [x={time}, y={freq}] {global_week.dat};
% \addlegendentry{Total}
\end{axis}
\end{tikzpicture}

    }
    \caption[Pattern of check-in during the week]{During the week, patterns
        are less clear.  \label{fig:weekly_checkin}}
    \end{subfigure}
    \caption{Check-ins temporal pattern.\label{fig:time_checkin}}
\end{figure}

\begin{figure}[hbt]
    \begin{subfigure}[b]{\textwidth}
    \centering
    \iftoggle{EXTERNALPGF}{\includegraphics[width=\textwidth]{day_cluster_3h}}{\input{mkfigs/day_cluster_3h}}
    \end{subfigure}

    \begin{subfigure}[b]{\textwidth}
    \centering
    \iftoggle{EXTERNALPGF}{\includegraphics[width=\textwidth]{day_cluster_4h}}{\input{mkfigs/day_cluster_4h}}
    \end{subfigure}
    \caption[Venues clustered by time of check-ins]{Venues clustered by time
    of check-ins. \label{fig:day_cluster}}
\end{figure}

\newgeometry{left=0.3cm,right=0.3cm,bottom=0.1cm,top=0.1cm}
\begin{center}
\begin{figure}[hbt]
    \begin{subfigure}[b]{0.4\textwidth}
    \centering
    \iftoggle{EXTERNALPGF}{\includegraphics[height=0.18\textheight]{cluster_day_3h_cl1}}{\begin{tikzpicture}
	\begin{axis}[%
            title={3 hours Cluster 1 in all cities},
            xlabel={hour},
            ylabel={percentage},
            grid=major,
            xticklabels={,1--4, 4--7, 7--10, 10--13, 13--16, 16--19, 19--22, 22--1},
		]
\addplot[ck1, thin, smooth] table[x index=0, y index=1] {amsterdam_cluster_day_3h.dat};
\addplot[ck2, thin, smooth,] table[x index=0, y index=1]{atlanta_cluster_day_3h.dat};
\addplot[ck3, thin, smooth,] table[x index=0, y index=1]{barcelona_cluster_day_3h.dat};
\addplot[ck4, thin, smooth,] table[x index=0, y index=1]{berlin_cluster_day_3h.dat};
\addplot[ck5, thin, smooth,] table[x index=0, y index=1]{chicago_cluster_day_3h.dat};
\addplot[ck6, thin, smooth,] table[x index=0, y index=1]{helsinki_cluster_day_3h.dat};
\addplot[ck7, thin, smooth,] table[x index=0, y index=1]{houston_cluster_day_3h.dat};
\addplot[ck8, thin, smooth,] table[x index=0, y index=1]{indianapolis_cluster_day_3h.dat};
\addplot[ck9, thin, smooth,] table[x index=0, y index=1]{london_cluster_day_3h.dat};
\addplot[ck10, thin, smooth,] table[x index=0, y index=1]{losangeles_cluster_day_3h.dat};
\addplot[ck11, thin, smooth,] table[x index=0, y index=1]{moscow_cluster_day_3h.dat};
\addplot[ck12, thin, smooth,] table[x index=0, y index=1]{newyork_cluster_day_3h.dat};
\addplot[ck13, thin, smooth,] table[x index=0, y index=1]{paris_cluster_day_3h.dat};
\addplot[ck14, thin, smooth,] table[x index=0, y index=1]{prague_cluster_day_3h.dat};
\addplot[ck15, thin, smooth,] table[x index=0, y index=1]{rome_cluster_day_3h.dat};
\addplot[ck16, thin, smooth,] table[x index=0, y index=1]{sanfrancisco_cluster_day_3h.dat};
\addplot[ck17, thin, smooth,] table[x index=0, y index=1]{seattle_cluster_day_3h.dat};
\addplot[ck18, thin, smooth,] table[x index=0, y index=1]{stlouis_cluster_day_3h.dat};
\addplot[ck19, thin, smooth,] table[x index=0, y index=1]{stockholm_cluster_day_3h.dat};
\addplot[ck20, thin, smooth,] table[x index=0, y index=1]{washington_cluster_day_3h.dat};
\end{axis}
\end{tikzpicture}
}
    \end{subfigure}~
    \begin{subfigure}[b]{0.4\textwidth}
    \centering
    \iftoggle{EXTERNALPGF}{\includegraphics[height=0.18\textheight]{cluster_day_4h_cl1}}{\begin{tikzpicture}
	\begin{axis}[%
            % title={ 4 hours Cluster 1 in all cities},
            % xlabel={hour},
            % ylabel={percentage},
		axis y line=none,
            grid=major,
            xticklabels={,2--5,5--9,9--13,13--17,17--21,21--2}
		]
\addplot[ck1,  smooth] table[x index=0, y index=1] {amsterdam_cluster_day_4h.dat};
\addplot[ck2,  smooth,] table[x index=0, y index=1]{atlanta_cluster_day_4h.dat};
\addplot[ck3,  smooth,] table[x index=0, y index=1]{barcelona_cluster_day_4h.dat};
\addplot[ck4,  smooth,] table[x index=0, y index=1]{berlin_cluster_day_4h.dat};
\addplot[ck5,  smooth,] table[x index=0, y index=1]{chicago_cluster_day_4h.dat};
\addplot[ck6,  smooth,] table[x index=0, y index=1]{helsinki_cluster_day_4h.dat};
\addplot[ck7,  smooth,] table[x index=0, y index=1]{houston_cluster_day_4h.dat};
\addplot[ck8,  smooth,] table[x index=0, y index=1]{indianapolis_cluster_day_4h.dat};
\addplot[ck9,  smooth,] table[x index=0, y index=1]{london_cluster_day_4h.dat};
\addplot[ck10,  smooth,] table[x index=0, y index=1]{losangeles_cluster_day_4h.dat};
\addplot[ck11,  smooth,] table[x index=0, y index=1]{moscow_cluster_day_4h.dat};
\addplot[ck12,  smooth,] table[x index=0, y index=1]{newyork_cluster_day_4h.dat};
\addplot[ck13,  smooth,] table[x index=0, y index=1]{paris_cluster_day_4h.dat};
\addplot[ck14,  smooth,] table[x index=0, y index=1]{prague_cluster_day_4h.dat};
\addplot[ck15,  smooth,] table[x index=0, y index=1]{rome_cluster_day_4h.dat};
\addplot[ck16,  smooth,] table[x index=0, y index=1]{sanfrancisco_cluster_day_4h.dat};
\addplot[ck17,  smooth,] table[x index=0, y index=1]{seattle_cluster_day_4h.dat};
\addplot[ck18,  smooth,] table[x index=0, y index=1]{stlouis_cluster_day_4h.dat};
\addplot[ck19,  smooth,] table[x index=0, y index=1]{stockholm_cluster_day_4h.dat};
\addplot[ck20,  smooth,] table[x index=0, y index=1]{washington_cluster_day_4h.dat};
\end{axis}
\end{tikzpicture}
}
    \end{subfigure}

    \begin{subfigure}[b]{0.4\textwidth}
    \centering
    \iftoggle{EXTERNALPGF}{\includegraphics[height=0.18\textheight]{cluster_day_3h_cl2}}{\input{mkfigs/cluster_day_3h_cl2}}
    \end{subfigure}~
    \begin{subfigure}[b]{0.4\textwidth}
    \centering
    \iftoggle{EXTERNALPGF}{\includegraphics[height=0.18\textheight]{cluster_day_4h_cl2}}{\begin{tikzpicture}
	\begin{axis}[%
            title={ 4 hours Cluster 2 in all cities},
            xlabel={hour},
            ylabel={percentage},
            grid=major,
            xticklabels={,2--5,5--9,9--13,13--17,17--21,21--2}
		]
\addplot[ck1, thin, smooth] table[x index=0, y index=2] {amsterdam_cluster_day_4h.dat};
\addplot[ck2, thin, smooth,] table[x index=0, y index=2]{atlanta_cluster_day_4h.dat};
\addplot[ck3, thin, smooth,] table[x index=0, y index=2]{barcelona_cluster_day_4h.dat};
\addplot[ck4, thin, smooth,] table[x index=0, y index=2]{berlin_cluster_day_4h.dat};
\addplot[ck5, thin, smooth,] table[x index=0, y index=2]{chicago_cluster_day_4h.dat};
\addplot[ck6, thin, smooth,] table[x index=0, y index=2]{helsinki_cluster_day_4h.dat};
\addplot[ck7, thin, smooth,] table[x index=0, y index=2]{houston_cluster_day_4h.dat};
\addplot[ck8, thin, smooth,] table[x index=0, y index=2]{indianapolis_cluster_day_4h.dat};
\addplot[ck9, thin, smooth,] table[x index=0, y index=2]{london_cluster_day_4h.dat};
\addplot[ck10, thin, smooth,] table[x index=0, y index=2]{losangeles_cluster_day_4h.dat};
\addplot[ck11, thin, smooth,] table[x index=0, y index=2]{moscow_cluster_day_4h.dat};
\addplot[ck12, thin, smooth,] table[x index=0, y index=2]{newyork_cluster_day_4h.dat};
\addplot[ck13, thin, smooth,] table[x index=0, y index=2]{paris_cluster_day_4h.dat};
\addplot[ck14, thin, smooth,] table[x index=0, y index=2]{prague_cluster_day_4h.dat};
\addplot[ck15, thin, smooth,] table[x index=0, y index=2]{rome_cluster_day_4h.dat};
\addplot[ck16, thin, smooth,] table[x index=0, y index=2]{sanfrancisco_cluster_day_4h.dat};
\addplot[ck17, thin, smooth,] table[x index=0, y index=2]{seattle_cluster_day_4h.dat};
\addplot[ck18, thin, smooth,] table[x index=0, y index=2]{stlouis_cluster_day_4h.dat};
\addplot[ck19, thin, smooth,] table[x index=0, y index=2]{stockholm_cluster_day_4h.dat};
\addplot[ck20, thin, smooth,] table[x index=0, y index=2]{washington_cluster_day_4h.dat};
\end{axis}
\end{tikzpicture}
}
    \end{subfigure}

    \begin{subfigure}[b]{0.4\textwidth}
    \centering
    \iftoggle{EXTERNALPGF}{\includegraphics[height=0.18\textheight]{cluster_day_3h_cl3}}{\begin{tikzpicture}
	\begin{axis}[%
            title={3 hours Cluster 3 in all cities},
            xlabel={hour},
            ylabel={percentage},
            grid=major,
            xticklabels={,1--4, 4--7, 7--10, 10--13, 13--16, 16--19, 19--22, 22--1},
		]
\addplot[ck1, thin, smooth] table[x index=0, y index=3] {amsterdam_cluster_day_3h.dat};
\addplot[ck2, thin, smooth,] table[x index=0, y index=3]{atlanta_cluster_day_3h.dat};
\addplot[ck3, thin, smooth,] table[x index=0, y index=3]{barcelona_cluster_day_3h.dat};
\addplot[ck4, thin, smooth,] table[x index=0, y index=3]{berlin_cluster_day_3h.dat};
\addplot[ck5, thin, smooth,] table[x index=0, y index=3]{chicago_cluster_day_3h.dat};
\addplot[ck6, thin, smooth,] table[x index=0, y index=3]{helsinki_cluster_day_3h.dat};
\addplot[ck7, thin, smooth,] table[x index=0, y index=3]{houston_cluster_day_3h.dat};
\addplot[ck8, thin, smooth,] table[x index=0, y index=3]{indianapolis_cluster_day_3h.dat};
\addplot[ck9, thin, smooth,] table[x index=0, y index=3]{london_cluster_day_3h.dat};
\addplot[ck10, thin, smooth,] table[x index=0, y index=3]{losangeles_cluster_day_3h.dat};
\addplot[ck11, thin, smooth,] table[x index=0, y index=3]{moscow_cluster_day_3h.dat};
\addplot[ck12, thin, smooth,] table[x index=0, y index=3]{newyork_cluster_day_3h.dat};
\addplot[ck13, thin, smooth,] table[x index=0, y index=3]{paris_cluster_day_3h.dat};
\addplot[ck14, thin, smooth,] table[x index=0, y index=3]{prague_cluster_day_3h.dat};
\addplot[ck15, thin, smooth,] table[x index=0, y index=3]{rome_cluster_day_3h.dat};
\addplot[ck16, thin, smooth,] table[x index=0, y index=3]{sanfrancisco_cluster_day_3h.dat};
\addplot[ck17, thin, smooth,] table[x index=0, y index=3]{seattle_cluster_day_3h.dat};
\addplot[ck18, thin, smooth,] table[x index=0, y index=3]{stlouis_cluster_day_3h.dat};
\addplot[ck19, thin, smooth,] table[x index=0, y index=3]{stockholm_cluster_day_3h.dat};
\addplot[ck20, thin, smooth,] table[x index=0, y index=3]{washington_cluster_day_3h.dat};
\end{axis}
\end{tikzpicture}
}
    \end{subfigure}~
    \begin{subfigure}[b]{0.4\textwidth}
    \centering
    \iftoggle{EXTERNALPGF}{\includegraphics[height=0.18\textheight]{cluster_day_4h_cl3}}{\input{mkfigs/cluster_day_4h_cl3}}
    \end{subfigure}

    \begin{subfigure}[b]{0.4\textwidth}
    \centering
    \iftoggle{EXTERNALPGF}{\includegraphics[height=0.18\textheight]{cluster_day_3h_cl4}}{\input{mkfigs/cluster_day_3h_cl4}}
    \end{subfigure}~
    \begin{subfigure}[b]{0.4\textwidth}
    \centering
    \iftoggle{EXTERNALPGF}{\includegraphics[height=0.18\textheight]{cluster_day_4h_cl4}}{\begin{tikzpicture}
	\begin{axis}[%
            title={ 4 hours Cluster 4 in all cities},
            xlabel={hour},
            ylabel={percentage},
            grid=major,
            xticklabels={,2--5,5--9,9--13,13--17,17--21,21--2}
		]
\addplot[ck1, thin, smooth] table[x index=0, y index=4] {amsterdam_cluster_day_4h.dat};
\addplot[ck2, thin, smooth,] table[x index=0, y index=4]{atlanta_cluster_day_4h.dat};
\addplot[ck3, thin, smooth,] table[x index=0, y index=4]{barcelona_cluster_day_4h.dat};
\addplot[ck4, thin, smooth,] table[x index=0, y index=4]{berlin_cluster_day_4h.dat};
\addplot[ck5, thin, smooth,] table[x index=0, y index=4]{chicago_cluster_day_4h.dat};
\addplot[ck6, thin, smooth,] table[x index=0, y index=4]{helsinki_cluster_day_4h.dat};
\addplot[ck7, thin, smooth,] table[x index=0, y index=4]{houston_cluster_day_4h.dat};
\addplot[ck8, thin, smooth,] table[x index=0, y index=4]{indianapolis_cluster_day_4h.dat};
\addplot[ck9, thin, smooth,] table[x index=0, y index=4]{london_cluster_day_4h.dat};
\addplot[ck10, thin, smooth,] table[x index=0, y index=4]{losangeles_cluster_day_4h.dat};
\addplot[ck11, thin, smooth,] table[x index=0, y index=4]{moscow_cluster_day_4h.dat};
\addplot[ck12, thin, smooth,] table[x index=0, y index=4]{newyork_cluster_day_4h.dat};
\addplot[ck13, thin, smooth,] table[x index=0, y index=4]{paris_cluster_day_4h.dat};
\addplot[ck14, thin, smooth,] table[x index=0, y index=4]{prague_cluster_day_4h.dat};
\addplot[ck15, thin, smooth,] table[x index=0, y index=4]{rome_cluster_day_4h.dat};
\addplot[ck16, thin, smooth,] table[x index=0, y index=4]{sanfrancisco_cluster_day_4h.dat};
\addplot[ck17, thin, smooth,] table[x index=0, y index=4]{seattle_cluster_day_4h.dat};
\addplot[ck18, thin, smooth,] table[x index=0, y index=4]{stlouis_cluster_day_4h.dat};
\addplot[ck19, thin, smooth,] table[x index=0, y index=4]{stockholm_cluster_day_4h.dat};
\addplot[ck20, thin, smooth,] table[x index=0, y index=4]{washington_cluster_day_4h.dat};
\end{axis}
\end{tikzpicture}
}
    \end{subfigure}

    \begin{subfigure}[b]{0.4\textwidth}
    \centering
    \iftoggle{EXTERNALPGF}{\includegraphics[height=0.18\textheight]{cluster_day_3h_cl5}}{\input{mkfigs/cluster_day_3h_cl5}}
    \end{subfigure}~
    \begin{subfigure}[b]{0.4\textwidth}
    \centering
    \iftoggle{EXTERNALPGF}{\includegraphics[height=0.18\textheight]{cluster_day_4h_cl5}}{\begin{tikzpicture}
	\begin{axis}[%
            title={ 4 hours Cluster 5 in all cities},
            xlabel={hour},
            ylabel={percentage},
            grid=major,
            xticklabels={,2--5,5--9,9--13,13--17,17--21,21--2}
		]
\addplot[ck1, thin, smooth] table[x index=0, y index=5] {amsterdam_cluster_day_4h.dat};
\addplot[ck2, thin, smooth,] table[x index=0, y index=5]{atlanta_cluster_day_4h.dat};
\addplot[ck3, thin, smooth,] table[x index=0, y index=5]{barcelona_cluster_day_4h.dat};
\addplot[ck4, thin, smooth,] table[x index=0, y index=5]{berlin_cluster_day_4h.dat};
\addplot[ck5, thin, smooth,] table[x index=0, y index=5]{chicago_cluster_day_4h.dat};
\addplot[ck6, thin, smooth,] table[x index=0, y index=5]{helsinki_cluster_day_4h.dat};
\addplot[ck7, thin, smooth,] table[x index=0, y index=5]{houston_cluster_day_4h.dat};
\addplot[ck8, thin, smooth,] table[x index=0, y index=5]{indianapolis_cluster_day_4h.dat};
\addplot[ck9, thin, smooth,] table[x index=0, y index=5]{london_cluster_day_4h.dat};
\addplot[ck10, thin, smooth,] table[x index=0, y index=5]{losangeles_cluster_day_4h.dat};
\addplot[ck11, thin, smooth,] table[x index=0, y index=5]{moscow_cluster_day_4h.dat};
\addplot[ck12, thin, smooth,] table[x index=0, y index=5]{newyork_cluster_day_4h.dat};
\addplot[ck13, thin, smooth,] table[x index=0, y index=5]{paris_cluster_day_4h.dat};
\addplot[ck14, thin, smooth,] table[x index=0, y index=5]{prague_cluster_day_4h.dat};
\addplot[ck15, thin, smooth,] table[x index=0, y index=5]{rome_cluster_day_4h.dat};
\addplot[ck16, thin, smooth,] table[x index=0, y index=5]{sanfrancisco_cluster_day_4h.dat};
\addplot[ck17, thin, smooth,] table[x index=0, y index=5]{seattle_cluster_day_4h.dat};
\addplot[ck18, thin, smooth,] table[x index=0, y index=5]{stlouis_cluster_day_4h.dat};
\addplot[ck19, thin, smooth,] table[x index=0, y index=5]{stockholm_cluster_day_4h.dat};
\addplot[ck20, thin, smooth,] table[x index=0, y index=5]{washington_cluster_day_4h.dat};
\end{axis}
\end{tikzpicture}
}
    \end{subfigure}
    \caption[Venues cluster by time among all the cities]{4 hours time
    clusters are the same among all cities, whereas it is not the case for 3
hours clusters. \label{fig:time_cluster_size}}
\end{figure}
\end{center}
\restoregeometry

\subsection{Space}

\begin{figure}[hbtp]
\includegraphics[width=\textwidth]{density_paris_venues}
\caption[Venue density in Paris]{Estimated venues density in Paris.
\label{fig:density_paris}}
\end{figure}

Density picture \autoref{fig:density_paris} remote versus central
locations

Explain categories surrounding.

\subsection{User entropy}

Explain why entropy is an interesting measure according to
\autocite{Entropy10}.

Let $\mathcal{U}$ be the set of users and $\mathcal{V}$ the one of venues. For
each venue $v \in \mathcal{V}$, we can gather the set of check-ins $c_v$ that
occurred there as well as the corresponding users $[u_1, u_2, u_2, \ldots,
u_n]$ (note that the same users can check-in multiple times). Then we
transform this list to a frequency distribution $f_v$ over $\mathcal{U}$ and
compute the normalized entropy: \[
    H(v) = -\frac{1}{\log\left(\left| \mathcal{U}\right|\right)}
\sum_{u\in \mathcal{U}} f_v(u) \log(f_v(u)) \]

Show example with high and low value in Paris and Barcelona:
\begin{tabular}{llr}
	\toprule
	Name                    & Category            & Entropy \\
	\midrule
	Castellers de Barcelona & Non-Profit          & 0.118064 \\
	Café de la Pompeu       & Café                & 0.146145 \\
	Ràdio 4                 & Radio Station       & 0.149995 \\
	La Comarca              & Home (private)      & 0.153078 \\
	House castellar         & Home (private)      & 0.154076 \\
	Av Tomas Gimenez        & Bar                 & 0.158411 \\
	Torre De Barad-Dur      & Building            & 0.163024 \\
	\midrule
	Parc de la Ciutadella   & Park                & 4.416021 \\
	Apple Store             & Electronics Store   & 4.560921 \\
	Parc Güell              & Park                & 4.815076 \\
	Plaça de Catalunya      & Plaza               & 4.977019 \\
	Sagrada Família         & Government Building & 5.356548 \\
	Sants Estació           & Train Station       & 5.360970 \\
	Camp Nou                & Stadium             & 5.829855 \\
	\bottomrule
\end{tabular}

Compute time entropy and see if there is a link. Not really, some place are
visited all the day by a very small group of people,
\autoref{fig:two_entropies}

\begin{figure}[hbtp]
\includegraphics[width=\textwidth]{two_entropies}
\caption{Venues entropy with respect to user population and time of check-in
during the day. \label{fig:two_entropies}}
\end{figure}


\subsection{Photos}

Assume that photos are background signal of interest within city. Use exact
grid algo from \autocite{Agarwal2006spatial} to compute discrepancy with
check-ins measurement (and conversely). \marginpar{give real examples} More
photos in touristic places, more check-in in “event” places (railway, stadium)

Look at high and low focus places: not so interesting either. High value are
for popular places like park or monument whereas low (but not zero) are for
mundane place (like pizzeria) close to famous locations.

\section{Chosen representation}

Based on the insights gained by exploring the dataset, we settle to represents
each venues by the numerical vector presented in \autoref{tab:venuefeatures}.
The features involving the surrounding (numbered from 6 to 16) were weighted
by a 2D Gaussian of radius $r=350$ meters.

\begin{table}[hb]
    \centering
    \begin{tabularx}{\textwidth}{lX}
        \toprule
        index & description \\
        \midrule
	\datasetRow{}0 & Number of likes \\
	\datasetRow{}1 & Number of unique users \\
	\datasetRow{}2 & Number of total check-ins \\
        3 & User entropy \\
        4 & Venue density \\
	\datasetRow{}5 & Venue top level category \\
        6 & \enquote{Arts \& Entertainment} venues around \\
        7 & \enquote{College \& University} venues around \\
        8 & \enquote{Food} venues around \\
        9 & \enquote{Nightlife Spot} venues around \\
        10 & \enquote{Outdoors \& Recreation} venues around \\
        11 & \enquote{Shop \& Service} venues around \\
        12 & \enquote{Professional \& Other Places} venues around \\
        13 & \enquote{Residence} venues around \\
        14 & \enquote{Travel \& Transport} venues around \\
	15 & Ratio of photos over check-ins \\
	16 & Ratio of photos associated to the venue over photos linked to other venues \\
	17 & More than half of the check-ins occurs during the week-end \\
	18 & Frequency of check-ins between 2 a.m. and 6 a.m. \\
	19 & Frequency of check-ins between 6 a.m. and 10 a.m. \\
	20 & Frequency of check-ins between 10 a.m. and 2 p.m. \\
	21 & Frequency of check-ins between 2 p.m. and 6 p.m. \\
	22 & Frequency of check-ins between 6 p.m. and 10 p.m. \\
	23 & Frequency of check-ins between 10 p.m. and 2 a.m. \\
        \bottomrule
    \end{tabularx}
    \caption[Venue features]{Feature representation of Foursquare venues,
	    taking surrounding into account. \colorbox{Moccasin}{Shaded
	    features} are provided directly by Foursquare database whereas the
	    others were computed solely based on the partial information
	    contained in the dataset.\label{tab:venuefeatures}}
\end{table}

\clearpage

\newgeometry{hmargin=1.1cm,vmargin=0.7cm}
\begin{table}[t]
	\centering
	\begin{tabularx}{\textwidth}{lXX}
		\toprule
		field & description & example \\
		\midrule
		\texttt{\_id}   & Flickr photo id                                    & 12683984675 \\
		\texttt{loc}    & GeoJSON location of the photo                      & {"type": "Point", "coordinates": [25.009667, 60.227586]} \\
		\texttt{uid}    & Flickr user id                                     & "40747751@N08" \\
		\texttt{taken}  & Local time of the shooting according to the camera & ISODate("2014-02-21T18:40:21Z") \\
		\texttt{title}  & User given title                                   & "One Way to Lock a Bicycle" \\
		\texttt{hint}   & city name                                          & "helsinki" \\
		\texttt{tags}   & Normalized user given tags                         & ["bicycle", "lock", "chain", padlock, "locked"] \\
		\texttt{venue}  & Foursquare venue id                                & null \\
		\texttt{upload} & UTC time at which the photo was uploaded           & ISODate("2014-02-22T00:28:32Z") \\
		\texttt{farm}   & Used to obtain a link to the picture file          & 6 \\
		\texttt{server} & Used to obtain a link to the picture file          & "5506" \\
		\texttt{secret} & Used to obtain a link to the picture file          & "74fbd5b60a" \\
		\bottomrule
	\end{tabularx}
	\caption[Photo format]{Description of the fields of the photo
	object.\label{tab:photo}}
\end{table}
\begin{table}[b]
    \centering
    \begin{tabularx}{\textwidth}{lXX}
        \toprule
        field & description & example \\
        \midrule
\texttt{\_id}          & Foursquare venue id                             & "4ba22d4cf964a5207ce137e3" \\
\texttt{name}          & Name of the venue                               & "Ravintola Oiva" \\
\texttt{loc}           & GeoJSON location according to Foursquare        & {"type" : "Point", "coordinates": [24.951011, 60.181716084480286]} \\
\texttt{cat}           & id of the primary category                      & "4bf58dd8d48988d11b941735" \\
\texttt{cats}          & Potential additional categories                 & ["4bf58dd8d48988d1c4941735", "4bf58dd8d48988d143941735"] \\
\texttt{checkinsCount} & Total number of check-ins in Foursquare         & \numprint{2556} \\
\texttt{usersCount}    & Number of unique user in Foursquare             & \numprint{1321} \\
\texttt{tipCount}      & Number of tip left by users                     & 13 \\
\texttt{price}         & Price range, from 1 to 4                        & 2 \\
\texttt{rating}        & Rating over 10                                  & 7.82 \\
\texttt{createdAt}     & Time at which the venue was added to Foursquare & ISODate("2010-03-18T15:40:28Z") \\
\texttt{mayor}         & User id of the mayor at the time of collection  & 18123276 \\
\texttt{tags}          & User contributed tags                           & ["karaoke"] \\
\texttt{shortUrl}      & Short URL appearing in tweets                   & "\url{http://4sq.com/bV8IRQ}" \\
\texttt{canonicalUrl}  & Full URL                                        & "\url{https://foursquare.com/v/ravintola-oiva/4ba22d4cf964a5207ce137e3}" \\
\texttt{likes}         & Total number of like in Foursquare              & 3 \\
\texttt{likers}        & List of at most ten users liking this venue     & [14695530, 9936722, 37114401] \\
\texttt{city}          & City name                                       & "helsinki" \\
\texttt{closed}        & True when Foursquare says the venue is closed   & null \\
\texttt{hours}         & Always \texttt{null}                            & null \\
        \bottomrule
    \end{tabularx}
    \caption[Venue format]{Description of the fields of the venue
    object.\label{tab:venue}}
\end{table}
\restoregeometry
