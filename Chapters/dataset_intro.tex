To solve the problem at hand, we extract information from data. Specifically,
we take of advantage of geo localized activities exposed on \fs{}
(\autoref{sec:foursquare}) and \flickr{} (\autoref{sec:flickr}). An extensive
exploration of the data collected (\autoref{sec:exploration}) leads us to
choose \emph{venue} as the elementary unit of information. Therefore, we
represent them by some numerical features (\autoref{sec:feature}) in the rest
of this work. Beside raw data, we also gathered neighborhoods ground truth by conducting a
small-scale user study (\autoref{sec:user-study}). Finally, as we are problem
and not data-driven, we conclude this chapter by assessing the adequacy of such
data toward our goal, and by presenting alternative sources of information
(\autoref{sec:limitations}).

To limit the scope of our study, we focus on 20 cities:
\begin{itemize}
	\item 10 located in the United States: Atlanta, Chicago, Houston,
		Indianapolis, Los Angeles, New York, Saint Louis, San Francisco,
		Seattle and Washington
	\item 10 located in Europe: Amsterdam, Barcelona, Berlin, Helsinki, London,
		Moscow, Paris, Prague, Rome and Stockholm
\end{itemize}

They were chosen for their high activity according to
\href{http://www.4sqstat.com/}{4sqstat.com}. We deliberately exclude other
parts of the world as we are not familiar enough with them to correctly
evaluate our results.

At the time of writing, this dataset is available at
\href{http://figshare.com/authors/G\%C3\%A9raud_Le_Falher/542931}%
{\url{figshare.com/authors/Géraud_Le_Falher/542931}}.
