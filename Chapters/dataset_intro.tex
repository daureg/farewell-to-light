There are many ways to obtain information about places. The most
straightforward is to ask directly people. For instance, \textcite{Curated14}
asked residents of Pittsburgh to build a city guide by sharing personal
stories about locations they know well. One can also mine text, like Yelp
reviews to discover shared topics among restaurant \autocite{YelpReview14}.
Other possibilities include Image analysis to describe place atmosphere
or mining GPS trajectories to discover which places are significant
\autocite{GPSStay10}

In this work, we choose to focus on traces of human activity. Namely we
collected Foursquare check-ins and Flickr photos metadata. The first step of
this process was to choose a fixed number of cities.

According to \href{http://www.4sqstat.com/}{4sqstat.com}, we first selected
the ten cities in the US with the highest number of check-ins: New York,
Washington, San Francisco, Atlanta, Indianapolis, Los Angeles, Seattle,
Houston, Saint Louis and Chicago. We followed a similar procedure in Europe,
resulting in the selection of London, Paris, Berlin, Rome, Prague, Moscow,
Amsterdam, Helsinki, Stockholm and Barcelona. This is admittedly a choice
biased by our own history and the ease to gather user input. Indeed, Rio de
Janeiro or Tokyo are more active than Helsinki or Stockholm, but less familiar
to us and to many students of Aalto University, making results more difficult
to interpret.

Besides their specific issues that we will explain later, these two datasets
suffers from general problem related to online sources. First, as in most of
psychological studies \autocite{Weird10}, the user distribution of online
services is biased toward young people, living in cities and with a high level
of life. Second, even among those, we observed an heavy tail phenomena; for
instance, 70\% of twitter users have tweeted less than 10 times
\href{http://twopcharts.com/twitteractivitymonitor}%
{\url{twopcharts.com/twitteractivitymonitor}})
