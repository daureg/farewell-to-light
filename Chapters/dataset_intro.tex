\chapter{Dataset}
\label{chap:dataset}

There are many ways to obtain information about places. The most
straightforward is to ask directly people. For instance, \textcite{Curated14}
\marginpar{describe better} asked residents of Pittsburgh to build a city
guide by sharing their opinion about locations they know well.
Mine text, like yelp review \autocite{YelpReview14}
Image analysis to describe place atmosphere
Transportation data to see which place are connected

In this work, we choose to focus on trace of human activity. Namely we
collected Foursquare check-ins and Flickr photos' metadata. The first step of
this process was to choose a fixed number of cities.

According to \href{http://www.4sqstat.com/}{4sqstat.com}, we first selected
the ten cities in the US with the highest number of check-ins: New York,
Washington, San Francisco, Atlanta, Indianapolis, Los Angeles, Seattle,
Houston, St. Louis and Chicago. We did something similar in Europe,
resulting in the selection of London, Paris, Berlin, Rome, Prague, Moscow,
Amsterdam, Helsinki, Stockholm and Barcelona. This is admittedly a choice
biased by our own history and the ease to gather user input. Indeed, Rio
de Janeiro or Tokyo are more active than Helsinki or Stockholm, but less
familiar to us and to many students of Aalto University, making results more
difficult to interpret.

Besides their own issues explained later, these two datasets suffers from
biased user distribution\autocite{Weird10}, heavy tail (half of twitter users
have never tweeted \href{http://twopcharts.com/twitteractivitymonitor}%
{\url{twopcharts.com/twitteractivitymonitor}}
