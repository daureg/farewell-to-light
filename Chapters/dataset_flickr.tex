\section{Flickr}

Using the Flickr API, we downloaded metadata\footnote{But no image at all.}
from every photo satisfying a set of criteria: they contained at least one
tag, they were located inside one of the chosen city and they have been
uploaded after January 1\textsuperscript{st}, 2008. This yields
\numprint{7748621} photos but as showed in \autoref{tab:dataset}, there are a
factor of 77 between the figures in New-York and Helsinki.

More precisely, in addition to tags and location, we know when each photos was
taken and uploaded, by which user and what title was given to it (the title
was not used except when it contained hashtags, which were converted to tags).
Thus a typical data point looks like what is described in \autoref{tab:photo}.

\subsection{Limitations}

This retrieval process was naturally not perfect. In addition to some API
calls returning strange results, the casual nature of the data explains their
inherent noise.

\begin{itemize}
	\item While timestamp issued by mobile phones are likely to be correct,
 as their internal clock is synchronized by internet, this may not
 always be the case for dedicated cameras. More concerning than usual
 drift of low quality clock is the situation of tourists coming from
 different timezone. Yet as we could not think of any simple solution to
 that problem, we just ignored it and carried on.
	\item To ensure the quality of the localization, we restricted results to
 photos whose precision is deemed \enquote{street level} by Flickr. The
 potential problem is that it would cost an extra request to know
 whether this location was given by GPS (in which case the camera
 position is accurate) or by the user at upload time. In the latter
 case, in addition to the general imprecision of the method, it is
 ambiguous whether this location refer the place where the photo was
 shot or the position of the photo's subject\footnote{Think of a bridge
 taken from a nearby hill.}.
	\item Finally, without additional request, the tags obtained are those
 normalized by Flickr. This normalization is not bijective but it is
 assumed that two tags with the same normalized form were close in the
 first place.
\end{itemize}
