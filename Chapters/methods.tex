\chapter{Methods}
\label{chap:methods}

\section{Dimensionality reduction}

\begin{flushright}{\slshape
 If people could see in high dimensions \\
 machine learning would not be necessary.
} \\ \medskip
--- Pedro \Textcite{MLKnowledge12}
\end{flushright}

A common idea in Machine Learning problems is that data are meaningful.
Although we do not perceive their structure in high dimensional space, we
assume that data are the result of a principled, underlying process and
therefore that sample points are not randomly distributed in the whole space.
A fruitful approach to validate this hypothesis is to perform dimensionality
reduction, with the aim of recovering a low dimensional manifold that embed
the data.

Another motivation is that matching would be easier in this low dimensional
manifold. We tried several standard method but the best performing was
t-Distributed Stochastic Neighbor Embedding \autocite{tSNE08}. Basically, it…

\section{Individual matching}

Roughly the idea was to use $k$-NN, first with basic euclidean distance and
later with more suitable metrics.

After reading some surveys \autocite{MetricSurvey06, MetricSurvey13}, I tried
Information-theoretic Metric Learning \autocite{InfoMetric07},
Gradient-Boosted Largest Margin Nearest Neighbors \autocite{GBLMNN12}. If I
have time, I would also like to try the recent Sparse Compositional Metric
Learning \autocite{SparseMetric14}, as its code has recently been made available.

Another point to mention is how results were evaluated. First by looking at
how far we need to go before finding a venue of the same brand. Second by
computed the Normalized Discounted Cumulative Gain over category
\autocite{IREvaluation07}.

\section{Neighborhood matching}

\subsection{Choosing a distance between set of points}
\subsection{Learning parameters from partial ground truth}
\subsection{Efficient search heuristic}

\section{Complete city}

\begin{comments}
Right now, I have not started this part at all. What I was initially thinking
is some sort of iterative process that alternate two steps: one where we
partition each city into neighborhoods, another where those neighborhoods are
matched. And the result of each step will influence the next one.

But it looks like time will be missing.
\end{comments}

\section{Visualization}
\begin{comments}
I do not think there will be a lot of time left for this part. Yet even if 2D
map are quite effective at showing results, it would have been nice to use
something in 3D. For instance, Nokia have pretty maps of some cities in the
dataset: \href{http://here.com/60.1670149,24.9477934,15.92,293,54,3d.day}%
{\url{http://here.com/60.1670149,24.9477934,15.92,293,54,3d.day}}
\end{comments}
