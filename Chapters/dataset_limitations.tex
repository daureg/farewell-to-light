\section{Limitations and other sources}
\label{sec:limitations}

These two datasets suffers from general problem related to online sources.
First, as its the case in most of psychological studies \autocite{Weird10}, the
user distribution of online services is biased toward young people, living in
cities and with a high level of life. Second, even among those, we observed an
heavy tail phenomena; for instance, 70\% of twitter users have tweeted less
than 10 times \href{http://twopcharts.com/twitteractivitymonitor}%
{\url{twopcharts.com/twitteractivitymonitor}}.

% According to \textcite{FSMotivation11}, the main factor are fun (because the
% application is gamified and its a lightweight activity to perform when one is
% bored), social aspect (see where your friends go and let them know what you
% are doing) and discovery (read tips left by previous users and get discount).

The data collection process described above is easily fooled by dishonest user
claiming to be at some locations even if it is not the case. According to
\autocite{FakeCheckins12}, this can be caused by the desire to \enquote{cheat}
the gaming component of many \glspl{lbsn}, take advantage of monetary
incentives offered by venues owner to go in their establishment or simply
provide allegedly objective alibi to cover for other activity. Solution
involving special hardware and protocol are not within our reach
\autocite{ValidateCheckin13} and even detecting such fake check-ins, by using
temporal tensor decomposition \autocite{FindingFake14} is a problem in itself.
Therefore we did not make any attempt at cleaning the data.

Another concern is the use of Twitter. First because we access only a sample
of all tweets, without any guarantee about its representativeness.
Fortunately, it does not appear that Twitter is purposely trying to trick
researchers \autocite{TwitterBias14}. Second because our tweets span more than
three years, a interval during which Twitter and its user behavior have
changed \autocite{TwitterEvolution14}.

We cannot be oblivious to privacy issues either.
Even if it would give us a more reliable picture, there are situations in
which people are reluctant (or unable) to check-in \autocite{Privacy11}. For
instance, when people are in hospitals, they have something else on their mind
than checking in and as a consequence, these places are not as well described
as restaurant or parks.
% \begin{comments}
% On the other hands some users are very open: One posted 5 tweets in less than
% one hour in a pub, along with 3 names of beer
% \href{https://untappd.com/user/fneri3/checkin/85857825}%
% {\url{untappd.com/user/fneri3/checkin/85857825}} and 2 photos
% \href{https://foursquare.com/fneri3/checkin/5373fb47498e65814631c3e3?s=D2ximdj1zhKVt_qZy9xpvObiHkc}%
% {\url{foursquare.com/fneri3/checkin/5373fb47498e65814631c3e3?s=D2ximdj1zhKVt_qZy9xpvObiHkc}}.
% \end{comments}

Finally, through this work, we relied on venues category provided by
Foursquare. Yet it has been showed recently than performing Latent Dirichlet
Allocation on associated text defined finer and more meaningful categorisation
\autocite{PlaceSemantic14}, which could benefit for the semi supervised
training described later.

\bigskip

\Textcite{TwitterMongoDB13} describe in great details a setup similar to ours.
But there are others ways to obtain information about places. The most
straightforward is to ask directly people. For instance, \textcite{Curated14}
asked residents of Pittsburgh to build a city guide by sharing personal stories
about locations they know well. One can also mine text, like Yelp reviews to
discover shared topics among restaurant \autocite{YelpReview14}.  Other
possibilities include image analysis to describe place atmosphere or mining GPS
trajectories to discover which places are significant \autocite{GPSStay10}.
