\section{Overview}
\label{sec:overview}

Our approach is briefly described as follows.  By collecting rich geo-enabled
data from social-media platforms, and by associating this data to the different
venues of the city, we can represent each venue by a feature vector, which
accurately describes the characteristics and the overall activity of the venue.
We then ask to devise similarity measures between venues, as well as between
neighborhoods, i.e., sets of venues that are geographically close to each
other. 

We address these two problems from a {\em metric-learning} point of view
\cite{MetricSurvey13}.  We experiment with many different distance measures
and with algorithms that aim to learn their parameters. %%  of the underlying
metric.  To learn the parameters of the distance measures and select the
optimal settings we use ground-truth, either obtained automatically with simple
rules or gathered from carefully-designed user studies. 

Our results indicate that the problem of finding a venue similar to a single
query venue is a challenging problem, as the accuracy of the various tasks we
design is relatively low.  This is probably due to the fact that data at the
venue level are noisy.  However, when aggregating many venues together and
asking to find similar neighborhoods, the results become of much higher
quality.  This is indicated in our case study in
Section~\ref{section:casestudy}.

The  measure that is shown to perform best for the task of finding similar
neighborhoods is the {\em earth-mover's distance} (\emd) \cite{EMD98}.  \emd\
is known to be a robust measure---however, it is also expensive to compute.
Motivated by this observation, we address the issue of computational
efficiency.  In particular,  given a neighborhood \region\ in one city, we ask
how to find the most similar neighborhood to \region\ in another city (or a set
of other cities) under \emd, and without performing brute force computation.
We design a pruning strategy that yields significant speed improvement with
minimal loss in accuracy. 

Our study and our algorithms are based on extensive experimental evaluation, on
data collected from many cities in Europe and in the US.  Our datasets consist
of activity logs gathered from \fs\ and \flickr, a location-based social
network and a photo-sharing platform, respectively.  Yet our study can be
enriched by many other types of data, such as transportation, weather, air
quality, energy consumption, etc.  Such an extension is left for future work.

Finally we offer some concluding remarks about the quality of our results,
about the overall work performed, and about future directions.

\section{Contributions}

\begin{itemize}
	\item Collect and compute relevant features describing venues.
	\item Seamlessly incorporate heterogeneous \emph{items} by defining
		appropriate metrics and learning their parameters.
	\item Efficient search in $\mathbb{R}^2$ instead of brute force for
		the query~\ref{q:space}.
	\item Compare results with existing methods and evaluate them against
		human judging.
\end{itemize}
