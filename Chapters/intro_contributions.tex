\subsection{Overview}
\subsection{Contributions}

\begin{itemize}
	\item Collect and compute relevant features describing venues.
	\item Seamlessly incorporate heterogeneous \emph{items} by defining
		appropriate metrics and learning their parameters
	\item Deliver interpretable results. Beyond a single number, the
		similarity, or absence thereof, must be motivated. It enables
		principled comparison of cities and more user friendly output
		(as well as justifying why no sensible result were found for
		some queries). Furthermore, this would address recent
		concerns about Algorithmic Accountability
		\autocite{Accountability13}. For instance,
		\textcite{Discrimination13} showed that in case of online
		advertisement discrimination, opacity of the algorithms makes
		it difficult to find and sanction those responsible.
        \begin{comments}
            Unfortunately, this is not really the case at the moment so maybe
            it will end up in future work. The easiest way to do it would be
            to find justification after the matching process (by clustering
            venues and summarizing it like \enquote{These two regions match
            because people take a lot of photos, go there on weekday from
            4pm to 8pm and there are a lot cultural buildings.} Yet it would
            be more satisfactory if the whole process was driven by such
            considerations.
        \end{comments}
	\item Efficient search in $\mathbb{R}^2$ instead of brute force for
		the query~\ref{q:space}.
	\item Compare results with existing methods and evaluate them against
		human judging.
\end{itemize}
