\section{Problem definition}
\label{sec:problem}

We have a set of cities $\mathcal{C} =
\{c_1, c_2, \ldots, c_n\}$, each of them defined by a enclosing rectangle
$R_{c_i} \subseteq \mathbb{R}^2$. Furthermore, each city $c_i$ contains a
subset $V_{c_i} \subseteq \mathcal{V}$, where $\mathcal{V}$ is a set of
\emph{venues}. A \emph{venue} is an uniquely identified location that can be
visited by individuals. It includes for instance restaurants, shops, airports,
monuments and parks. Finally, each city $c_i$ is described by another subset
$I_{c_i} \subseteq \mathcal{I}$, where $\mathcal{I}$ is a set of \emph{items}.
These \emph{items} are user contributed discrete points in space and time that
can have additional attributes. Here, we exploit Flickr photos (with tags) and
Foursquare check-ins as type of item\footnote{But this definition can be
    extended to other kind of data. For instance, a tweet is a point in time
    and space (along with the sentiment it conveys). We can also consider
    noise and pollution measurements, the number of cars crossing a given
    intersection during a one minute window, Yelp reviews, etc. Stretching the
    concept, we could collect anonymized customer receipts from grocery shops.
    The attributes will then be the proportion in which different kind of
    goods are consumed. Indeed, even for two shops of the same company in the
    same city, it would give information about the neighboring population.}.
Items are associated with venues, either directly, like check-ins, or
indirectly, like photos. In the latter case, the relationship between venues
and items is based on spatial distance.

By combining venues own characteristics and their associated items, we want to
devise a similarity measure over the venues $\mathcal{V}$, namely $\delta:
\mathcal{V}^2 \rightarrow \llbracket 0, 1 \rrbracket $. More generally, we want
another measure of similarity between two spatial regions. That is, after
choosing a home city $c_h$ and a target city $c_t$\footnote{$t$ can also stand
for travel or tourist.}, we want to construct $\delta': R_{c_h} \times R_{c_t}
\rightarrow \llbracket 0, 1 \rrbracket$. These two functions answer the
following queries:

\begin{enumerate}
\item When users pick a venue $v$ in their home city, find the venue in the
	target town that is the most similar to it. Formally, we compute
	$\argmax_{v' \in V_{c_t}} \delta(v, v')$.\label{q:point}
\item Instead of a single venue, users ask for a larger area $r$, for instance
	covering a neighborhood\footnote{or any other meaningful subdivision.}
	from their home city. In that case we also return an area:
	$\argmax_{r' \subseteq V_{c_t}} \delta'(r, r')$.\label{q:space}
\end{enumerate}


\begin{comments}
another way to say it:\\
We have two cities, $C_1$ and $C_2$. In each of them lie objects of different
kind (in our case: photos, check-ins and venues). These objects have a
position in $\mathbb{R}^2$, along with other attributes like time, tags,
category or number of visitors.

As an input, we are given a polygonal region $R$ contained in $C_1$. The
desired output is the region $R'$ in $C_2$ that is closest to $R$ according
to the metric $\delta_{\theta}$.

\end{comments}

Because the second problem has wider scope, let us provide more details.
A region is represented by the $e'$ main objects it contains (the venues).
Each of these venues is in turn described by $d$ numerical features. In
addition, each region have a set of $g$ global features that are not
associated with any specific objects. Thus a region $R$ is fully characterized
by
\begin{align*}
	\phi \colon \mathcal{R} &\to
	\mathbb{R}^{e'\times d} \times \mathbb{R}^g = \mathcal{F} \\
	R &\mapsto (X, x)
\end{align*}

Inside $\mathcal{F}$, two regions can be compared using the
$\theta$-parametrized distance
\begin{align*}
	\delta_{\theta} \colon &\mathcal{F} \times \mathcal{F} \to
	\mathbb{R}^+ \\
	&(X, x) , (X', x') \mapsto d_{R,R'}
\end{align*}

and as mentioned, our our objective is fulfilled by computing, for a given $R
\in \mathcal{R}(C_1)$
\begin{equation}
	\argmin_{R' \in \mathcal{R}(C_2)}\; \delta_{\theta}
	\left(\phi(R), \phi(R')\right)
	\label{e:onetheta}
\end{equation}

More precisely, $\delta_{\theta}$ is defined as follow.\marginpar{Actually,
that's only one of the possible definition.} Let $X_i$ (respectively
$X'_i$) be the $i$\textsuperscript{th} column of $X$ ($X'$), $P_i$ ($Q_i$) the
associated distribution, $D_{\mathrm{KL}}(P \parallel Q) = \sum_j
P(j)\ln\left(\frac{P(j)}{Q(j)}\right)$ the Kullback--Leibler divergence
between $P$ and $Q$ and $ \mathrm{JSD}(P \parallel Q)=
\frac{1}{2}D_{\mathrm{KL}}(P \parallel M)+\frac{1}{2}D_{\mathrm{KL}}(Q
\parallel M)$ their Jensen--Shannon divergence \autocite{JensenShannon03},
where $M=\frac{1}{2}(P+Q)$. Then
\[
	\delta_{\theta} \left(\phi(R), \phi(R')\right) = \sum_{i=1}^d
	\theta_i \mathrm{JSD}(P_i \parallel Q_i) + \vnorm{x - x'}
\]

\subsection*{Additional considerations}
\begin{comments}
    It originally went for more than one page but the main points were:
    \begin{itemize}
        \item $\theta$ can varies between query, which is more flexible.
        \item this can be formulated as Point Pattern Matching problem.
        \item Venues can be weighted before comparison.
        \item There are three kind of distances: geometric, probabilistic and
            graph based.
        \item We can combine different distances.
    \end{itemize}
\end{comments}
