\section{Problem definition and notation}
\label{sec:problem}

Here we describe our data and their context abstractly before articulating them
into a specific problem.

\subsection{Cities, venues and activities}
We consider a set \cities\ of $\noc$ cities, 
$\cities=\{\city_1, \city_2, \ldots,\city_\noc\}$.
Each city $\city$ contains a set of \emph{venues} $\venues(\city)$.
A venue is an uniquely identified location that can be visited by individuals.
It includes for example restaurants, shops, airports, monuments and parks.
We denote the set of all venues by $\allvenues = \bigcup_{\city\in\cities} \venues(\city)$.

Each city \city{} is also described by another subset $A_\city \subseteq
\activities$, where \activities{} is a set of \emph{activities}.  These
activities are user contributed discrete points in space and time that can hold
additional attributes. Here, we exploit Flickr photos (with tags) and
Foursquare check-ins as type of item.

In general, many other types of activity can be considered.  Geo-located tweets
fit our model directly, as a tweet is composed by a user, has spatio-temporal
coordinates, and contains text. The latter may conveys sentiment, which can be
extracted via sentiment-analysis methods \autocite{SentimentAnalysis08}.
Reviews collected from Yelp or other reviewing platforms also fit our model
nicely.  We could also consider noise and pollution measurements or traffic
flow for cars, pedestrians, and bikes, measured by city sensors or mobile
devices.  Stretching the concept, one could even consider collecting anonymized
customer receipts (\eg{}, grocery shops, restaurants, or other services) and
record the distribution of services provided and goods consumed in a city.

Activities are associated with venues, either explicitly, in the case of
check-ins, or implicitly, as with photos. In the latter case, the relationship
between venues and activity is based on spatial distance.  By combining venues
own characteristics and their associated items, we represent them by a {\em
feature vector} $\vfv(\venue)$ (see \autoref{sec:feature}
\vpageref{sec:feature} for a complete description).


\subsection{Distance between venues}
Given two venues $\venue_i,\venue_j\in\allvenues$ with 
feature vectors $\vfv_i = \vfv(\venue_i)$ and $\vfv_j = \vfv(\venue_j)$, 
the simplest distance one can define between the two venues is
the $p$-norm between their feature vectors $||\vfv_i-\vfv_j||_p$.
This simple definition has a number of shortcomings, as it does not
account for 
($i$) the different scale of different features, 
($ii$) the different scale of the same feature due to variations in the cities, 
($iii$) the importance of features, and 
($iv$) potential correlations between features.
One way to address the first two shortcomings is to ensure that each feature,
aggregated at a city level, has mean equal to zero and standard deviation equal
to one. 

To address the other two shortcomings, we can introduce a parametrization
$\theta$ of usual distances.  Learning this parameter $\theta$, and thus the
distance measure between venues, is the first challenge we face:
\begin{objective}
\label{objective:venue-distance-learning}
Learn the parameter $\theta$ so that the distance measure $\vdist_\theta$
captures best the human perception of similarity between venues.  
\end{objective} 
The theoretical motivation for metric learning is exposed in
\autoref{sub:metric-learning} and our
own results are described in \autoref{sec:venues-metrics}.

\subsection{City neighborhoods}
Summarizing our discussion so far, each venue $\venue\in\allvenues$ is
described by the pair $(\vloc(\venue), \vfv(\venue))$, where $\vloc(\venue)$
and $\vfv(\venue)$ specify the location of the venue~\venue, and its feature
vector, respectively. 

Now, given a city $\city\in\cities$, we define a {\em neighborhood} 
(or a {\em region}) \region\ of the city \city\ as a geographical region
(a closed and connected set in the geographical plane).
We abuse notation and we write $\region\subseteq\city$ to denote that
\region\ is a neighborhood of the city~\city.
For a neighborhood $\region\subseteq\city$ we define $\venues(\region)$ to
be the set of venues of \city\ that are
contained in \region, i.e., 
\begin{equation*}
\venues(\region) = \{ \venue\in\allvenues 
\mid \vloc(\venue)\in\region \}.  
\end{equation*}

We also define $\vfvs(\region)$ to be the set of feature vectors of
all the venues in \region, that is, 
\begin{equation*}
 \vfvs(\region) = \{ \vfv(\venue) \mid \vloc(\venue)\in\region \}.  
\end{equation*}


\subsection{Distance between neighborhoods}
Our objective is to define a meaningful distance measure between city
neighborhoods.
We want two neighborhoods to be similar, 
if they host comparable activities.
Thus, given two neighborhoods $\region_i$ and $\region_j$ 
we consider the feature vectors $\vfvs(\region_i)$ and
$\vfvs(\region_j)$ of the venues contained the two neighborhoods, and 
we define the distance $\rdist(\region_i,\region_j)$ between
$\region_i$ and $\region_j$ by
\begin{equation}
\label{eq:neighborhood-distance}
\rdist(\region_i,\region_j) = 
\sdist(\vfvs(\region_i),\vfvs(\region_j)),
\end{equation}
where \sdist\ is a distance function between 
{\em sets of feature vectors}. 
Our second objective is stated as follows.
\begin{objective}
\label{objective:neighborhood-distance-learning}
Design a distance measure \rdist\ between city neighborhoods as
expressed by \eqref{eq:neighborhood-distance},
\ie{} a distance measure \sdist\ between sets of feature vectors  of the
venues in the two neighborhoods. 
The distance measure should capture best the human
perception of similarity between city neighborhoods. 
\end{objective}
In \autoref{sec:regions-metrics}
we consider a number of different options for the distance function
\sdist, and we discuss our methodology for selecting the best one.
Building on the optimal distance measure selected for our objective, 
we then consider the following {\em city-neighborhood search} problem. 

\begin{problem}
\label{problem:neighborhood-search}
We are given a neighborhood \region\ in a city $\city\in\cities$
and a subset of target cities $\cities'\subseteq\cities$.
The goal is to find a neighborhood $\region'$ in some city 
$\city'\in\cities'$ so that the distance
$\rdist(\region,\region')$ is minimized. 
\end{problem}

Two interesting special cases of
Problem~\ref{problem:neighborhood-search} are
($i$)  $\cities'=\cities$, 
search for the most similar neighborhood in all cities; 
and 
($ii$) $\cities'=\{C'\}$, 
search for the most similar neighborhood in a given city $\city'$.
The emphasis for Problem~\ref{problem:neighborhood-search} is on computational
efficiency, since given a distance function \rdist{}, one can always apply a
brute-force search.  We discuss our solution in Section~\ref{ch:approx}.
